\section{Introduction}

%{\em The introduction should briefly place the study in a broad context and highlight why it is important. It should define the purpose of the work and its significance. The current state of the research field should be reviewed carefully and key publications cited. Please highlight controversial and diverging hypotheses when necessary. Finally, briefly mention the main aim of the work and highlight the principal conclusions.}

\subsection{Motivation}

Power quality is not currently a concern for most people in developed nations. Just like most people in developed nations assume that their tap water is of adequate quality to drink, most also assume that their electricity is of adequate quality to power their homes and appliances without causing harm. And, in both cases, most people assume that public utilities will appropriately monitor and correct any quality problems if they occur.

Successfully maintaining adequate power quality and providing sufficient amounts of it to meet the rising needs of consumers has been a triumph of electrical utilities for over 100 years. In recent times, however, there have been changes to the nature of electrical generation and consumption that make power quality of increasing concern and interest. First, there is a global need to shift to renewable energy sources. Second, traditional approaches to top-down grid power quality monitoring may not be well suited to bottom-up, distributed generation associated with renewable energy sources such as rooftop photovoltaic panels (PV). Third, modern consumer electronics place more stringent demands on power quality.  Finally, effective public policy making in this modern context can be aided by public access to power quality data.  Let's look at each of these in a bit more detail.

{\em 1. We need more renewable energy.} Concerns including pollution, environmental degradation, and climate change have produced a global movement away from centralized, fossil-fuel based forms of electrical energy generation and toward distributed, renewable alternatives such as wind and solar. But the economic, environmental, and political advantages of renewable energy comes with significant new technical challenges. Wind and solar are intermittent (for example, solar energy cannot be harvested at night) and unpredictable (for example, wind and solar energy fluctuate based upon cloud cover and wind speed). In addition, renewable energy generation is often distributed throughout the grid (such as in the case of residential rooftop photovoltaic (PV) systems).

One impact of adding renewable energy generation to an electrical grid is that maintaining adequate power quality is much more challenging. This problem is inversely proportional to the size of the electrical grid. For example, each island in the State of Hawaii maintains its own electrical grid, ranging from a 1.8 GW grid for the island of Oahu to a 6 MW grid for the island of Molokai.  (In contrast, the size of the European electrical grid is approximately 600 GW and the size of the U.S. continental grid is over 1000 GW.) For small grids, the unpredictable nature of power generation by distributed, intermittent renewables can quickly create problems for grid stability. In the case of Hawaii, consumer demand for grid-tied rooftop solar exceeded the ability of Hawaiian Electric to manage, resulting first in complex and expensive "interconnection studies" \cite{trabish_solar_2014,anastasi_energy_2009}, and later in the requirement for new installations to include batteries and grid-disconnected operation.

{\em 2. Traditional top-down monitoring is not well-suited to bottom-up energy generation.} For traditional grid architectures where generation is centralized and under the complete control of the utility, it is common to monitor power quality only to the substation level, because it can be assumed that the power quality experienced at the substation is a reasonably accurate proxy for the power quality experienced by the 1,000 or so end-users serviced by the substation. Furthermore, if an end-user experiences a power quality problem not experienced by the substation, then it is mostly likely due to equipment or electrical issues local to that end-user and not a grid-level problem.

These assumptions might not hold in grids with distributed, intermittent generation by end-users, such as is the case with roof top solar. In these cases, the unpredictable nature of generation can lead to local power quality problems that are not experienced by the substation, and that are not due to any single individual, but rather the bottom-up power generation architecture of the grid. Nakafuji notes that Hawaiian Electric is contending with PV penetrations in excess of 60\% on certain distribution circuits, for which traditional rules of thumb for design of protection and distribution systems might not hold \cite{nakafuji_back--basics_2011}.

{\em 3. Consumer electronics require higher quality power.} The rise of consumer electronics has raised the bar for what constitutes ``adequate" power quality. Only a few decades ago, computers were a rare presence outside of labs and large institutions. Today, computers are everywhere: embedded in phones, washers, refrigerators, thermostats, and so forth. These electronic devices not only have higher power quality requirements, but some of them actually introduce power quality problems in the form of harmonic distortion. Poor power quality can result in electronic devices failing unpredictably, and/or decreasing their lifespan.

{\em 4. Power quality data should be publicly available.} Electrical utilities are not required to be totally transparent about the quality of power they provide to consumers. For example, in Hawaii, utilities are only required to make a "best effort" to provide non-harmful voltage and frequency values, and are only required to publicly report about power outages of more than 3 minutes. There is no requirement for utilities to report potentially harmful levels of voltage, frequency, THD, or transients as long as they do not lead to outages. Indeed, in places like Hawaii, there is not yet even infrastructure in place to enable utilities to collect that information, even if they were asked to report on it.  Providing a high quality, third party, publicly available source of power quality data not only benefits the public, it makes it much easier to perform research on ways to improve grid stability in the presence of renewable energy generation.

Providing power quality data to address the above issues requires solving several difficult technical problems. First, monitoring below the substation level increases the number of required monitoring devices by two to three orders of magnitude. This has significant implications for the cost of monitoring in terms of the devices, and the sheer amount of low-level power quality data that will be produced. Second, this low-level data must be processed in a manner that does not require inordinate amounts of processing, network bandwidth, or storage. Finally, collecting the data is not useful if it cannot be converted into actionable information in a reasonable amount of time.

\subsection{Goals of the Open Power Quality Project}

The goal of the OPQ Project is to provide a scalable source of actionable power quality data about electrical grids, particularly those with high levels of distributed, intermittent power generation. OPQ accomplishes this through the design and implementation of a low-cost sensor network for power quality using custom power quality monitors with real-time, two way communication to a set of cloud-based services. An OPQ sensor network is designed to be useful to: consumers who want to better understand the performance of their public utilities; utilities who desire a low-cost, easily deployable and re-deployable sensor network to gather information about grid stability that complements their existing infrastructure; to researchers who wish to design improvements to grid management and need a low cost mechanism for measuring the impact of their improvements; and to public policy makers responsible for designing and implementing regulatory frameworks for electrical utilities.  Several key features of our project emerge from these goals.

{\em OPQ Box.} First, our custom power quality monitoring hardware device, called "OPQ Box" can be produced for approximately US\$75. This is from 10 to 100 times less expensive than commercial power quality monitors. This cost differential means, for example, that instead of monitoring power quality at the level of individual buildings, OPQ makes it economically feasible to monitor power quality on each floor or even in each room of a building. Similarly, instead of monitoring power quality at the substation level of the grid, OPQ Boxes make it feasible to deploy dozens or hundreds into a community to obtain fine-grained data about the impact of solar or other renewables as directly perceived by the customer.

{\em Cloud-native information architecture.} Second, our information architecture is "cloud native", which means that OPQ Boxes are not designed for stand-alone, autonomous use, unlike current commercially available power quality monitors. Instead, our hardware boxes and cloud-based services perform real-time, two-way communication over the Internet to determine what power quality data must be gathered and with what fidelity in order to most efficiently produce actionable information. Two-way communication between the OPQ Boxes and could services creates the ability to control individual OPQ Boxes based upon the global state of all OPQ Boxes in an OPQ sensor network. As a result, OPQ cloud services can obtain and analyze high fidelity wave form data only when desired, without the prohibitive network overhead of constantly communicating this data from boxes to the cloud. This enables OPQ installations to be simultaneously responsive and scalable.

{\em Plug-in analysis architecture.} Third, all OPQ software services are designed for extensibility and interoperation. Our middleware components (Mauka and Makai) provide a plugin architecture. Our visualization component (View) is built from modular UI elements using React.  This means that OPQ is not a monolithic, closed system with a frozen feature set, but rather a design environment for experimentation with advanced classification and analysis of power quality data.

{\em Open Source.} Fourth, our software and hardware designs and implementations are made available using open source licenses. This means that organizations choosing OPQ do not face the business risk of committing resources to a single vendor with proprietary hardware and software. Our goal is to facilitate the creation of a community of researchers and industry practitioners to replicate, extend, and apply the insights gained from the OPQ system.

While we believe OPQ offers a compelling combination of capabilities, we want to be clear that all sensor network designs involve trade-offs. As we will show below, other commercial and research solutions have features not provided by OPQ that may be important depending upon an organization's specific needs. In some cases, OPQ may not be the appropriate choice. In other cases, a hybrid solution consisting of OPQ in combination with other technologies for grid monitoring and analysis may be most appropriate.

\subsection{Structure of this paper}

The remainder of this paper is organized as follows. Sections \ref{sec:system-architecture} and \ref{sec:information-architecture} present an overview of the OPQ system architecture and information architecture. Section \ref{sec:related-work} presents related research and technology.  Section \ref{sec:opq-box} discusses OPQ Box and Section \ref{sec:opq-makai} its triggering system. Section \ref{sec:opq-mauka} describes our analysis services, and Section \ref{sec:opq-view} our user interface.  Section \ref{sec:pilot} discusses the pilot deployment of a sensor network and its results. Section \ref{sec:conclusions} concludes with some proposals for future research.


