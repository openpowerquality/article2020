\section{Introduction}

%{\em The introduction should briefly place the study in a broad context and highlight why it is important. It should define the purpose of the work and its significance. The current state of the research field should be reviewed carefully and key publications cited. Please highlight controversial and diverging hypotheses when necessary. Finally, briefly mention the main aim of the work and highlight the principal conclusions.}

Power quality is not currently a concern for most people in developed nations. Just like most people in developed nations assume that their tap water is of adequate quality to drink, most also assume that their electricity is of adequate quality to power their homes and appliances without causing harm. And, in both cases, most people assume that public utilities will appropriately monitor and correct any quality problems if they occur.

Successfully maintaining adequate power quality and providing sufficient amounts of electrical power to meet the rising needs of consumers has been a triumph of electrical utilities for over 100 years. In recent times, however, there have been changes to the nature of electrical generation and consumption that make power quality of increasing concern and interest. First, there is a global need to shift to renewable energy sources. Second, traditional approaches to top-down grid power quality monitoring may not be well suited to bottom-up, distributed generation associated with renewable energy sources such as rooftop photovoltaic panels (PV). Third, modern consumer electronics place more stringent demands on power quality.  Finally, effective public policy making in this modern context can be aided by increased public access to power quality data reflecting the state of the grid as it is experienced by end-users.  Let's examine each of these assertions in more detail.

{\em 1. We need more renewable energy.} Concerns including pollution, environmental degradation, and climate change have produced a global movement away from centralized, fossil-fuel based forms of electrical energy generation and toward distributed, renewable alternatives such as wind and solar. But the economic, environmental, and political advantages of renewable energy come with significant new technical challenges. Wind and solar are intermittent (for example, solar energy cannot be harvested at night) and unpredictable (for example, wind and solar energy fluctuate based upon cloud cover and wind speed). In addition, renewable energy generation is often distributed throughout the grid (such as in the case of residential rooftop photovoltaic (PV) systems).

One impact of adding renewable energy generation to an electrical grid is that maintaining adequate power quality is much more challenging. This problem is inversely proportional to the size of the electrical grid. For example, each island in the State of Hawaii maintains its own electrical grid, ranging from a 1.8 GW grid for the island of Oahu to a 6 MW grid for the island of Molokai.  (In contrast, the size of the European electrical grid is approximately 600 GW and the size of the U.S. continental grid is over 1000 GW.) For small grids, the unpredictable nature of power generation by distributed, intermittent renewables can quickly create problems for grid stability. In the case of Hawaii, consumer demand for grid-tied rooftop solar exceeded the ability of Hawaiian Electric to manage, resulting first in complex and expensive "interconnection studies" \cite{trabish_solar_2014,anastasi_energy_2009}, and later in the requirement for new installations to include batteries and grid-disconnected operation. Photovoltaics affect power quality by introducing harmonics \cite{anurangi_effects_2017}. The power feed-in of PV generation in rural low-voltage grids can influence power quality (PQ) as well as facility operation and reliability \cite{rita_pinto_impact_2016}. In fact, a study of the island of Porto Santo in Portugal found that intermittent nature of installed photovoltaic and wind energy could result in a potential drop in frequency of 12 Hz, lasting as long as 7 seconds \cite{delgado_solutions_2011}.

{\em 2. Traditional top-down monitoring is not well-suited to bottom-up energy generation.} For traditional grid architectures where generation is centralized and under the complete control of the utility, it is common to monitor power quality only to the substation level, because it can be assumed that the power quality experienced at the substation is a reasonably accurate proxy for the power quality experienced by the 1,000 or so end-users serviced by the substation. Furthermore, if an end-user experiences a power quality problem not experienced by the substation, then it is mostly likely due to equipment or electrical issues local to that end-user and not a grid-level problem.

These assumptions might not hold in grids with distributed, intermittent generation by end-users, such as is the case with roof top solar. In these cases, the unpredictable nature of generation can lead to local power quality problems that are not experienced by the substation, and that are not due to any single individual, but rather the bottom-up power generation architecture of the grid. Solar panels connected to low voltage networks will result in over-voltages, the switching frequency of the converters in wind turbines causes high-frequency signals flowing into the grid, and harmonics are generated by EV chargers. \cite{zavoda_power_2018}
 Nakafuji notes that Hawaiian Electric is contending with PV penetrations in excess of 60\% on certain distribution circuits, for which traditional rules of thumb for design of protection and distribution systems might not hold \cite{nakafuji_back--basics_2011}.

{\em 3. Consumer electronics require high quality power.} The rise of consumer electronics has raised the bar for what constitutes ``adequate" power quality. Only a few decades ago, computers were a rare presence outside of labs and large institutions. Today, computers are everywhere: embedded in phones, washers, refrigerators, thermostats, and so forth. These electronic devices not only have higher power quality requirements, but some of them actually introduce power quality problems in the form of harmonic distortion. Poor power quality can result in electronic devices failing unpredictably, and/or decreasing their lifespan. Reduced input voltage can cause excessive power supply heat dissipation, resulting in reduced mean time between failures (MTBF). In addition, rectifiers and DC-to-DC converter switching transistors draw high-peak currents, which raise their junction temperatures. These temperature excursions take a toll on semiconductor longevity. High input voltage can also puncture a power supply's rectifier and switching transistor junctions, causing MTBF reduction and eventual breakdown. High-voltage transients lasting microseconds can permanently wreck the power supply and its electronic equipment load. \cite{dedad_when_2008}.

{\em 4. Power quality data should be publicly available.} Electrical utilities are not required to be totally transparent about the quality of power they provide to consumers. For example, in Hawaii, utilities are only required to make a "best effort" to provide non-harmful voltage and frequency values, and are only required to publicly report about power outages of more than 3 minutes. There is no requirement for utilities to report potentially harmful levels of voltage, frequency, THD, or transients as long as they do not lead to outages. Indeed, in places like Hawaii, there is not yet even infrastructure in place to enable utilities to collect that information, even if they were asked to report on it.  Providing a high quality, third party, publicly available source of power quality data not only benefits the public, it makes it much easier to perform research on ways to improve grid stability in the presence of renewable energy generation.

Better public understanding of power quality is important because it is such a significant economic problem. Across all business sectors, the U.S. economy is estimated to be losing between \$104 billion and \$164 billion a year to outages and another \$15 billion to \$24 billion to power quality phenomena \cite{elphick_summary_2015}. India lost more than \$9.6 billion in 2008 due to power quality problems, and Europe is estimated to be losing \$150 billion per year \cite{laskar_power_2012}.

Providing power quality data to address the above issues requires solving several difficult technical problems. First, monitoring below the substation level increases the number of required monitoring devices by two to three orders of magnitude. This has significant implications for the cost of monitoring in terms of the devices, and the sheer amount of low-level power quality data that will be produced. Second, this low-level data must be processed in a manner that does not require inordinate amounts of processing, network bandwidth, or storage. Finally, collecting the data is not useful if it cannot be converted into actionable information in a reasonable amount of time. This paper describes results to date of the Open Power Quality project, which attempts to address these problems.
