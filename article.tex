%  LaTeX support: latex@mdpi.com
%  In case you need support, please attach all files that are necessary for compiling as well as the log file, and specify the details of your LaTeX setup (which operating system and LaTeX version / tools you are using).

%=================================================================
% \documentclass[energies,article,submit,moreauthors,pdftex]{Definitions/mdpi}
\documentclass[preprints,article,accept,moreauthors,pdftex]{Definitions/mdpi}

%=================================================================
\firstpage{1}
\makeatletter
\setcounter{page}{\@firstpage}
\makeatother
\pubvolume{xx}
\issuenum{1}
\articlenumber{5}
\pubyear{2020}
\copyrightyear{2020}
%\externaleditor{Academic Editor: name}
\history{Received: date; Accepted: date; Published: date}
%\updates{yes} % If there is an update available, un-comment this line

%% MDPI internal command: uncomment if new journal that already uses continuous page numbers
%\continuouspages{yes}


%=================================================================
% Add packages and commands here. The following packages are loaded in our class file: fontenc, inputenc, calc, indentfirst, fancyhdr, graphicx,epstopdf, lastpage, ifthen, lineno, float, amsmath, setspace, enumitem, mathpazo, booktabs, titlesec, etoolbox, tabto, xcolor, soul, multirow, microtype, tikz, totcount, amsthm, hyphenat, natbib, hyperref, footmisc, url, geometry, newfloat, caption

%=================================================================
%% Please use the following mathematics environments: Theorem, Lemma, Corollary, Proposition, Characterization, Property, Problem, Example, ExamplesandDefinitions, Hypothesis, Remark, Definition, Notation, Assumption
%% For proofs, please use the proof environment (the amsthm package is loaded by the MDPI class).

%=================================================================
% Full title of the paper (Capitalized)
\Title{Design, Implementation, and Evaluation of Open Power Quality}

% Authors, for the paper (add full first names)
\Author{Anthony J. Christe, Sergey Negrashov and Philip M. Johnson}

% Authors, for metadata in PDF
\AuthorNames{Anthony J. Christe, Sergey Negrashov and Philip M. Johnson}

% Affiliations / Addresses (Add [1] after \address if there is only one affiliation.)
\address{%
Department of Information and Computer Sciences\\
University of Hawaii at Manoa\\
Honolulu, HI 96822\\
johnson@hawaii.edu}

% Abstract (single paragraph, 200 words max, background, methods, results, conclusions. Currently 193 words.)

\abstract{Modern electrical grids are transitioning from a centralized generation architecture to an architecture which must accomodate distributed, intermittent generation.  This transition also means that the formerly sharp distinction between energy producers (i.e. utility companies) and consumers (residences, businesses, etc) are blurring: residences can now both produce and consume energy, making energy policy more complex. The Open Power Quality (OPQ) project began in 2013 with the goal of designing and implementing a low cost, distributed power quality sensor network in order to provide make useful information about electrical grids available to producers, consumers, and policy makers.  Since then, we have designed low cost hardware devices that monitor power quality  and low-cost cloud-based software services that can economically analyze the data and detect a variety of anomalies. In 2019, we performed a pilot study where 15 OPQ hardware devices were deployed across the University of Hawaii microgrid for three months.  Results of the pilot study provide evidence that OPQ provides a variety of useful monitoring services and that the system could be scaled to service larger geographic regions. We conclude that OPQ provides a new and useful approach to power quality monitoring.}

% Keywords
\keyword{Power Quality, Open Source, Renewable Energy, Grid Stability}


\begin{document}

%The order of the section titles is: Introduction, Materials and Methods, Results, Discussion, Conclusions for these journals: aerospace,algorithms,antibodies,antioxidants,atmosphere,axioms,biomedicines,carbon,crystals,designs,diagnostics,environments,fermentation,fluids,forests,fractalfract,informatics,information,inventions,jfmk,jrfm,lubricants,neonatalscreening,neuroglia,particles,pharmaceutics,polymers,processes,technologies,viruses,vision

\section{Introduction}

%{\em The introduction should briefly place the study in a broad context and highlight why it is important. It should define the purpose of the work and its significance. The current state of the research field should be reviewed carefully and key publications cited. Please highlight controversial and diverging hypotheses when necessary. Finally, briefly mention the main aim of the work and highlight the principal conclusions.}

Power quality is not currently a concern for most people in developed nations. Just like most people in developed nations assume that their tap water is of adequate quality to drink, most also assume that their electricity is of adequate quality to power their homes and appliances without causing harm. And, in both cases, most people assume that public utilities will monitor and correct any quality problems if they occur.



Successfully maintaining adequate power quality and providing sufficient amounts of it to meet the rising needs of consumers has been a triumph of electrical utilities for over 100 years. In recent times, however, there have been changes to the nature of electrical generation and consumption that make power quality of increasing public concern and interest. First, there is a global need to shift to renewable energy sources. Second, modern consumer electronics place more stringent demands on power quality.  Third, effective policy making in this modern context requires better public access to power quality data.

{\em We need more renewable energy.} There is now a global movement away from centralized, fossil-fuel based forms of electrical energy generation and toward distributed, renewable alternatives such as wind and solar. But the economic, environmental, and political advantages of renewable energy comes with significant new technical challenges. Wind and solar are intermittent (for example, solar energy cannot be harvested at night) and unpredictable (for example, wind and solar energy fluctuate based upon cloud cover and wind speed). In addition, renewable energy generation can be distributed throughout the grid (such as in the case of residential rooftop photovoltaic (PV) systems).

One impact of adding renewable energy generation to an electrical grid is that maintaining adequate power quality is much more challenging. This problem is inversely proportional to the size of the electrical grid. For example, in Hawaii on the island of Oahu, the public demand for rooftop solar exceeds Hawaiian Electric Company's ability to utilize it and maintain adequate power quality. As a result, in many neighborhoods, new rooftop solar installation is prohibited because Hawaiian Electric fears its impact on not only local power quality but global grid stability \cite{TBD}.

{\em Consumer electronics require higher quality power.} The rise of consumer electronics has raised the bar for what constitutes ``adequate" power quality. Only a few decades ago, computers were a rare presence outside of labs and large institutions. Today, computers are everywhere: embedded in phones, washers, refrigerators, thermostats, and so forth. These electronic devices not only have higher power quality requirements, but some of them actually introduce power quality problems in the form of harmonic distortion. Poor power quality can result in electronic devices failing unpredictably, and/or decreasing their lifespan.

{\em Power quality data should be publicly available.} Electrical utilities are not required to be totally transparent about the quality of power they provide to consumers. For example, in Hawaii, utilities are required to make a "best effort" to provide non-harmful voltage and frequency values, but are only required to publicly report on power outages of more than 3 minutes. There is no requirement for utilities to report potentially harmful forms of voltage, frequency, THD, or transients. Indeed, in Hawaii, there is not yet infrastructure in place that would enable utilities to collect that information, even if they were asked to report on it. For more details, see Hawaiian Electric Tariff Rule No. 2 Character of Service and Hawaiian Electric Tariff Rule No. 16 Interruption of Service.

OPQ intends to provide an unbiased, independent, third party source of accurate power quality data. This can be used by consumers to better understand the performance of their public utilities, by researchers to devise improvements to grid control, and by public policy makers responsible for designing and implementing regulatory frameworks for electrical utilities.  The OPQ Project has produced a suite of innovations that can help address each of these issues.

{\em OPQ Box.} First, our power quality monitoring hardware device, called "OPQ Box" can be produced for approximately US\$75. This is from 10 to 100 times less expensive than commercial power quality monitors. This means, for example, that instead of monitoring power quality at the level of individual buildings, OPQ makes it economically feasible to monitor power quality on each floor or even in each room of a building. Similarly, instead of monitoring power quality at the substation level of the grid, OPQ Boxes make it feasible to deploy dozens or hundreds into a community to obtain fine-grained data about the impact of solar or other renewables as directly perceived by the customer.

{\em Cloud native.} Second, our architecture is "cloud native", which means that OPQ Boxes are not designed for "stand alone", autonomous use. Instead, our hardware boxes and cloud-based services perform real-time, two-way communication over the Internet to determine what power quality data must be gathered and analyzed. This creates the ability to control individual OPQ Boxes based upon the global state of all OPQ Boxes. It means our cloud services can obtain and analyze high fidelity wave form data only when desired, without the prohibitive network overhead of constantly communicating this data from boxes to the cloud. This enables OPQ installations to be simultaneously responsive and scalable.

{\em Extensibility.} Third, all OPQ software subsystems are designed for extensibility and interoperation. Our middleware components (Mauka and Makai) provide a plugin architecture. Our visualization component (View) is built from modular UI elements using React. We are also developing API endpoints for interoperation with other software systems. This means that OPQ is not a monolithic, closed system with a frozen feature set, but rather a design environment for experimentation with advanced classification and analysis of power quality data.

{\em Open Source.} Fourth, our software and hardware designs and implementations are made available using open source licenses. This means that organizations choosing OPQ do not face the business risk of committing resources to a single vendor with proprietary hardware and software. Our goal is to facilitate the creation of a community of researchers and industry practitioners to replicate, extend, and apply the insights gained from the OPQ system.

Finally, while we believe OPQ offers a compelling combination of capabilities, we want to be clear that all designs involve trade-offs. As we will show below, other commercial and research solutions have features not provided by OPQ that may be important depending upon an organization's specific needs. In some cases, OPQ may not be the appropriate choice. In other cases, a hybrid solution consisting of OPQ in combination with other technologies for power monitoring may be most appropriate. Our goal in this paper is to describe OPQ well enough for organizations to make a thoughtful decision about the use of our technology.





%%%%%%%%%%%%%%%%%%%%%%%%%%%%%%%%%%%%%%%%%%
\section{Results}

This section may be divided by subheadings. It should provide a concise and precise description of the experimental results, their interpretation as well as the experimental conclusions that can be drawn.
\begin{quote}
This section may be divided by subheadings. It should provide a concise and precise description of the experimental results, their interpretation as well as the experimental conclusions that can be drawn.
\end{quote}

%%%%%%%%%%%%%%%%%%%%%%%%%%%%%%%%%%%%%%%%%%
\subsection{Subsection}
\unskip
\subsubsection{Subsubsection}

Bulleted lists look like this:
\begin{itemize}[leftmargin=*,labelsep=5.8mm]
\item	First bullet
\item	Second bullet
\item	Third bullet
\end{itemize}

Numbered lists can be added as follows:
\begin{enumerate}[leftmargin=*,labelsep=4.9mm]
\item	First item
\item	Second item
\item	Third item
\end{enumerate}

The text continues here.

\subsection{Figures, Tables and Schemes}

All figures and tables should be cited in the main text as Figure 1, Table 1, etc.

\begin{figure}[H]
\centering
\includegraphics[width=2 cm]{Definitions/logo-mdpi}
\caption{This is a figure, Schemes follow the same formatting. If there are multiple panels, they should be listed as: (\textbf{a}) Description of what is contained in the first panel. (\textbf{b}) Description of what is contained in the second panel. Figures should be placed in the main text near to the first time they are cited. A caption on a single line should be centered.}
\end{figure}

Text

Text

\begin{table}[H]
\caption{This is a table caption. Tables should be placed in the main text near to the first time they are cited.}
\centering
%% \tablesize{} %% You can specify the fontsize here, e.g., \tablesize{\footnotesize}. If commented out \small will be used.
\begin{tabular}{ccc}
\toprule
\textbf{Title 1}	& \textbf{Title 2}	& \textbf{Title 3}\\
\midrule
entry 1		& data			& data\\
entry 2		& data			& data\\
\bottomrule
\end{tabular}
\end{table}

Text

Text

%\begin{listing}[H]
%\caption{Title of the listing}
%\rule{\textwidth}{1pt}
%\raggedright Text of the listing. In font size footnotesize, small, or normalsize. Preferred format: left aligned and single spaced. Preferred border format: top border line and bottom border line.
%\rule{\textwidth}{1pt}
%\end{listing}


\subsection{Formatting of Mathematical Components}

This is an example of an equation:

\begin{equation}
a + b = c
\end{equation}
%% If the documentclass option "submit" is chosen, please insert a blank line before and after any math environment (equation and eqnarray environments). This ensures correct linenumbering. The blank line should be removed when the documentclass option is changed to "accept" because the text following an equation should not be a new paragraph.

Please punctuate equations as regular text. Theorem-type environments (including propositions, lemmas, corollaries etc.) can be formatted as follows:
%% Example of a theorem:
\begin{Theorem}
Example text of a theorem.
\end{Theorem}

The text continues here. Proofs must be formatted as follows:

%% Example of a proof:
\begin{proof}[Proof of Theorem 1]
Text of the proof. Note that the phrase `of Theorem 1' is optional if it is clear which theorem is being referred to.
\end{proof}
The text continues here.

%%%%%%%%%%%%%%%%%%%%%%%%%%%%%%%%%%%%%%%%%%
\section{Discussion}

Authors should discuss the results and how they can be interpreted in perspective of previous studies and of the working hypotheses. The findings and their implications should be discussed in the broadest context possible. Future research directions may also be highlighted.

%%%%%%%%%%%%%%%%%%%%%%%%%%%%%%%%%%%%%%%%%%
\section{Materials and Methods}

Materials and Methods should be described with sufficient details to allow others to replicate and build on published results. Please note that publication of your manuscript implicates that you must make all materials, data, computer code, and protocols associated with the publication available to readers. Please disclose at the submission stage any restrictions on the availability of materials or information. New methods and protocols should be described in detail while well-established methods can be briefly described and appropriately cited.

Research manuscripts reporting large datasets that are deposited in a publicly available database should specify where the data have been deposited and provide the relevant accession numbers. If the accession numbers have not yet been obtained at the time of submission, please state that they will be provided during review. They must be provided prior to publication.

Interventionary studies involving animals or humans, and other studies require ethical approval must list the authority that provided approval and the corresponding ethical approval code.

%%%%%%%%%%%%%%%%%%%%%%%%%%%%%%%%%%%%%%%%%%
\section{Conclusions}

This section is not mandatory, but can be added to the manuscript if the discussion is unusually long or complex.

%%%%%%%%%%%%%%%%%%%%%%%%%%%%%%%%%%%%%%%%%%
\section{Patents}
This section is not mandatory, but may be added if there are patents resulting from the work reported in this manuscript.

%%%%%%%%%%%%%%%%%%%%%%%%%%%%%%%%%%%%%%%%%%
\vspace{6pt}

%%%%%%%%%%%%%%%%%%%%%%%%%%%%%%%%%%%%%%%%%%
%% optional
%\supplementary{The following are available online at \linksupplementary{s1}, Figure S1: title, Table S1: title, Video S1: title.}

% Only for the journal Methods and Protocols:
% If you wish to submit a video article, please do so with any other supplementary material.
% \supplementary{The following are available at \linksupplementary{s1}, Figure S1: title, Table S1: title, Video S1: title. A supporting video article is available at doi: link.}

%%%%%%%%%%%%%%%%%%%%%%%%%%%%%%%%%%%%%%%%%%
\authorcontributions{For research articles with several authors, a short paragraph specifying their individual contributions must be provided. The following statements should be used ``Conceptualization, X.X. and Y.Y.; methodology, X.X.; software, X.X.; validation, X.X., Y.Y. and Z.Z.; formal analysis, X.X.; investigation, X.X.; resources, X.X.; data curation, X.X.; writing--original draft preparation, X.X.; writing--review and editing, X.X.; visualization, X.X.; supervision, X.X.; project administration, X.X.; funding acquisition, Y.Y. All authors have read and agreed to the published version of the manuscript.'', please turn to the  \href{http://img.mdpi.org/data/contributor-role-instruction.pdf}{CRediT taxonomy} for the term explanation. Authorship must be limited to those who have contributed substantially to the work reported.}

%%%%%%%%%%%%%%%%%%%%%%%%%%%%%%%%%%%%%%%%%%
\funding{Please add: ``This research received no external funding'' or ``This research was funded by NAME OF FUNDER grant number XXX.'' and  and ``The APC was funded by XXX''. Check carefully that the details given are accurate and use the standard spelling of funding agency names at \url{https://search.crossref.org/funding}, any errors may affect your future funding.}

%%%%%%%%%%%%%%%%%%%%%%%%%%%%%%%%%%%%%%%%%%
\acknowledgments{In this section you can acknowledge any support given which is not covered by the author contribution or funding sections. This may include administrative and technical support, or donations in kind (e.g., materials used for experiments).}

%%%%%%%%%%%%%%%%%%%%%%%%%%%%%%%%%%%%%%%%%%
\conflictsofinterest{Declare conflicts of interest or state ``The authors declare no conflict of interest.'' Authors must identify and declare any personal circumstances or interest that may be perceived as inappropriately influencing the representation or interpretation of reported research results. Any role of the funders in the design of the study; in the collection, analyses or interpretation of data; in the writing of the manuscript, or in the decision to publish the results must be declared in this section. If there is no role, please state ``The funders had no role in the design of the study; in the collection, analyses, or interpretation of data; in the writing of the manuscript, or in the decision to publish the results''.}

%%%%%%%%%%%%%%%%%%%%%%%%%%%%%%%%%%%%%%%%%%
%% optional
\abbreviations{The following abbreviations are used in this manuscript:\\

\noindent
\begin{tabular}{@{}ll}
MDPI & Multidisciplinary Digital Publishing Institute\\
DOAJ & Directory of open access journals\\
TLA & Three letter acronym\\
LD & linear dichroism
\end{tabular}}

%%%%%%%%%%%%%%%%%%%%%%%%%%%%%%%%%%%%%%%%%%
%% optional
\appendixtitles{no} % Leave argument "no" if all appendix headings stay EMPTY (then no dot is printed after "Appendix A"). If the appendix sections contain a heading then change the argument to "yes".
\appendix
\section{}
\unskip
\subsection{}
The appendix is an optional section that can contain details and data supplemental to the main text. For example, explanations of experimental details that would disrupt the flow of the main text, but nonetheless remain crucial to understanding and reproducing the research shown; figures of replicates for experiments of which representative data is shown in the main text can be added here if brief, or as Supplementary data. Mathematical proofs of results not central to the paper can be added as an appendix.

\section{}
All appendix sections must be cited in the main text. In the appendixes, Figures, Tables, etc. should be labeled starting with `A', e.g., Figure A1, Figure A2, etc.

%%%%%%%%%%%%%%%%%%%%%%%%%%%%%%%%%%%%%%%%%%
\reftitle{References}

% Please provide either the correct journal abbreviation (e.g. according to the “List of Title Word Abbreviations” http://www.issn.org/services/online-services/access-to-the-ltwa/) or the full name of the journal.
% Citations and References in Supplementary files are permitted provided that they also appear in the reference list here.

%=====================================
% References, variant A: external bibliography
%=====================================
%\externalbibliography{yes}
%\bibliography{your_external_BibTeX_file}

%=====================================
% References, variant B: internal bibliography
%=====================================
\begin{thebibliography}{999}
% Reference 1
\bibitem[Author1(year)]{ref-journal}
Author1, T. The title of the cited article. {\em Journal Abbreviation} {\bf 2008}, {\em 10}, 142--149.
% Reference 2
\bibitem[Author2(year)]{ref-book}
Author2, L. The title of the cited contribution. In {\em The Book Title}; Editor1, F., Editor2, A., Eds.; Publishing House: City, Country, 2007; pp. 32--58.
\end{thebibliography}

% The following MDPI journals use author-date citation: Arts, Econometrics, Economies, Genealogy, Humanities, IJFS, JRFM, Laws, Religions, Risks, Social Sciences. For those journals, please follow the formatting guidelines on http://www.mdpi.com/authors/references
% To cite two works by the same author: \citeauthor{ref-journal-1a} (\citeyear{ref-journal-1a}, \citeyear{ref-journal-1b}). This produces: Whittaker (1967, 1975)
% To cite two works by the same author with specific pages: \citeauthor{ref-journal-3a} (\citeyear{ref-journal-3a}, p. 328; \citeyear{ref-journal-3b}, p.475). This produces: Wong (1999, p. 328; 2000, p. 475)


%%%%%%%%%%%%%%%%%%%%%%%%%%%%%%%%%%%%%%%%%%
%% optional
\sampleavailability{Samples of the compounds ...... are available from the authors.}

%% for journal Sci
%\reviewreports{\\
%Reviewer 1 comments and authors’ response\\
%Reviewer 2 comments and authors’ response\\
%Reviewer 3 comments and authors’ response
%}

%%%%%%%%%%%%%%%%%%%%%%%%%%%%%%%%%%%%%%%%%%
\end{document}

