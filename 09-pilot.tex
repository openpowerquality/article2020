\section{Results from the Pilot Deployment}
\label{sec:pilot-study}

To evaluate the capabilities of the OPQ Sensor Network, 15 OPQ Boxes were deployed at the University of Hawaii Manoa campus over the course of three months in the Fall of 2019.  This location was advantageous because it is an isolated microgrid connected to the Oahu powergrid only via a single 46kV feeder as shown in Figure 4.5. Another advantage of the UH campus is the high number of smart meters deployed across various levels of the power delivery infrastructure. While the purpose of these meters is to monitor the power consumption, they do include some rudimentary power quality monitoring capabilities. Data from the campus deployed meters was used as ground truth for comparison against the measurements, and for analysis performed by the OPQ project. The location of smart meters in the grid topology is shown in Figure 4.5 as the M nodes. As evident by the meter location none of them were monitoring the consumer level power and mainly focused on the higher voltage power delivery. This placement was a consequence of the smart meters’ role as a consumption monitor, and thus the deployment of the OPQ Boxes at the residential level complemented UH power quality monitoring capabilities without introducing redundancies.
                                                                                                                                    The University of Hawaii power grid supplies a highly diverse infrastructure. Beyond traditional residential equipment such as computers and consumer grade electronics, the UH power grid powers scientific and laboratory equipment, machine shops, and server farms. All of these elements have varying requirements/tolerances for power quality anomalies as well as different levels of power quality “pollution”. Furthermore, some of the electricity consumers in the UH campus are entirely unique. For example, the free electron laser located in the Watanabe Hall is one of the only free electron lasers in the world, and the impact/sensitivity of power quality on the instrument are completely unstudied.

\subsection{OPQ Boxes provide valid and reliable collection of power quality data}

\subsection{The OPQ Information Architecture provides a means to produce actionable insights}

\subsection{The OPQ Information Architecture provides predictable upper bounds on storage resources}

\subsection{OPQ Mauka provides useful adaptive optimization capabilities}



