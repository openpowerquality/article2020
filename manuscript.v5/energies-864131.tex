% LaTeX support: latex@mdpi.com
% In case you need support, please attach all files that are necessary for compiling as well as the log file, and specify the details of your LaTeX setup (which operating system and LaTeX version / tools you are using).

%=================================================================
\documentclass[energies,article,accept,moreauthors,pdftex]{Definitions/mdpi}
% \documentclass[preprints,article,accept,moreauthors,pdftex]{Definitions/mdpi}

%=================================================================
\firstpage{1}
\makeatletter
\setcounter{page}{\@firstpage}
\makeatother
\pubvolume{xx}
\issuenum{1}
\articlenumber{5}
\pubyear{2020}
\copyrightyear{2020}
%\externaleditor{Academic Editor: name}
\history{Received: 27 June 2020; Accepted: 28 July 2020; Published: date}
\updates{yes} % If there is an update available, un-comment this line

%% MDPI internal command: uncomment if new journal that already uses continuous page numbers
%\continuouspages{yes}


%=================================================================
% Add packages and commands here. The following packages are loaded in our class file: fontenc, inputenc, calc, indentfirst, fancyhdr, graphicx,epstopdf, lastpage, ifthen, lineno, float, amsmath, setspace, enumitem, mathpazo, booktabs, titlesec, etoolbox, tabto, xcolor, soul, multirow, microtype, tikz, totcount, amsthm, hyphenat, natbib, hyperref, footmisc, url, geometry, newfloat, caption

\usepackage[labelformat=simple]{subcaption}
\renewcommand\thesubfigure{\alph{subfigure}}
\DeclareCaptionLabelFormat{subcaptionlabel}{\normalfont(\textbf{#2}\normalfont)}
\captionsetup[subfigure]{labelformat=subcaptionlabel}

\usepackage{tcolorbox}
\usepackage{tabularx}

%=================================================================
%% Please use the following mathematics environments: Theorem, Lemma, Corollary, Proposition, Characterization, Property, Problem, Example, ExamplesandDefinitions, Hypothesis, Remark, Definition, Notation, Assumption
%% For proofs, please use the proof environment (the amsthm package is loaded by the MDPI class).

%=================================================================
% Full title of the paper (Capitalized)
\Title{%The paper was slightly modified by our English editor, please check the whole text and confirm if your meaning is retained; 2.	Do not delete any comment we left for you and reply to each comment so that we can understand your meaning clearly; 3.	If you need to revise somewhere in your paper, please highlight the revisions to let us know. (Thank you for your cooperation in advance.)  (CONFIRMED)
\highlight{Design}, Implementation, and Evaluation of Open Power Quality}

% Authors, for the paper (add full first names)
\Author{\hl{Anthony} %Please carefully check the accuracy of names and affiliations. (CONFIRMED)
J. Christe, Sergey Negrashov and Philip M. Johnson *}

% Authors, for metadata in PDF
\AuthorNames{Anthony J. Christe, Sergey Negrashov and Philip M. Johnson}

% Affiliations / Addresses (Add [1] after \address if there is only one affiliation.)
\address[1]{%
Department of Information and Computer Sciences, University of Hawaii at Manoa, Honolulu, HI 96822, USA; achriste@hawaii.edu (A.J.C.); sin8@hawaii.edu (S.N.)
}

\corres{Correspondence: johnson@hawaii.edu}


% Abstract (single paragraph, 200 words max, background, methods, results, conclusions. Currently 193 words.)

\abstract{Modern electrical grids are transitioning from a centralized generation architecture to an architecture with significant distributed, intermittent generation. This transition means that the formerly sharp distinction between energy producers (utility companies) and consumers (residences, businesses, etc.) are blurring: end-users both produce and consume energy, making energy management and public policy more complex. The goal of the Open Power Quality (OPQ) project is to design and implement a low cost, distributed power quality sensor network that provides useful new forms of information about modern electrical grids to producers, consumers, researchers, and~policy makers. In 2019, we performed a pilot study involving the deployment of an OPQ sensor network at the University of Hawaii microgrid for three months. Results of the pilot study validate the ability of OPQ to collect accurate power quality data in a way that provides useful new insights into electrical grids.}

% Keywords
\keyword{power quality; open source; renewable energy; grid stability}


\begin{document}

\section{Introduction}

%{\em The introduction should briefly place the study in a broad context and highlight why it is important. It should define the purpose of the work and its significance. The current state of the research field should be reviewed carefully and key publications cited. Please highlight controversial and diverging hypotheses when necessary. Finally, briefly mention the main aim of the work and highlight the principal conclusions.}

Power quality is not currently a concern for most people in developed nations. Just like most people in developed nations assume that their tap water is of adequate quality to drink, most also assume that their electricity is of adequate quality to power their homes and appliances without causing harm. In~both cases, most people assume that public utilities will appropriately monitor and correct any quality problems if they~occur.

Successfully maintaining adequate power quality and providing sufficient amounts of electrical power to meet the rising needs of consumers has been a triumph of electrical utilities for over 100~years. In~recent times, however, there have been changes to the nature of electrical generation and consumption that make power quality of increasing concern and interest. First, there is a global need to shift to renewable energy sources. Second, traditional approaches to top-down grid power quality monitoring may not be well suited to bottom-up, distributed generation associated with renewable energy sources such as rooftop photovoltaic panels (PV). Third, modern consumer electronics place more stringent demands on power quality. Finally, effective public policy making in this modern context can be aided by increased public access to power quality data reflecting the state of the grid as it is experienced by end-users. Let us examine each of these assertions in more~detail.

\highlight{{\em 1. We need more renewable energy.}} %Is italics necessary?
 Concerns including pollution, environmental degradation, and~climate change have produced a global movement away from centralized, fossil-fuel-based forms of electrical energy generation and toward distributed, renewable alternatives such as wind and solar. However,~the economic, environmental, and~political advantages of renewable energy come with significant new technical challenges. Wind and solar are intermittent (for example, solar energy cannot be harvested at night) and unpredictable (for example, wind and solar energy fluctuate based upon cloud cover and wind speed). In~addition, renewable energy generation is often distributed throughout the grid (such as in the case of residential rooftop photovoltaic (PV) systems).

One impact of adding renewable energy generation to an electrical grid is that maintaining adequate power quality is much more challenging. This problem is inversely proportional to the size of the electrical grid. For~example, each island in the State of Hawaii maintains its own electrical grid, ranging from a 1.8 GW grid for the island of Oahu to a 6 MW grid for the island of Molokai. (In contrast, the~size of the European electrical grid is approximately 600 GW and the size of the U.S. continental grid is over 1000 GW.) For small grids, the~unpredictable nature of power generation by distributed, intermittent renewables can quickly create problems for grid stability. In~the case of Hawaii, consumer demand for grid-tied rooftop solar exceeded the ability of Hawaiian Electric to manage, resulting first in complex and expensive ``interconnection studies''~\cite{trabish_solar_2014,anastasi_energy_2009}, and~later in the requirement for new installations to include batteries and grid-disconnected operation. Photovoltaics affect power quality by introducing harmonics~\cite{anurangi_effects_2017}. The~power feed-in of PV generation in rural low-voltage grids can influence power quality (PQ) as well as facility operation and reliability~\cite{rita_pinto_impact_2016}. In~fact, a~study of the island of Porto Santo in Portugal found that the intermittent nature of installed photovoltaic and wind energy could result in a potential drop in frequency of 12 Hz, lasting as long as 7 s~\cite{delgado_solutions_2011}.

\highlight{{\em 2. Traditional top-down monitoring is not well-suited to bottom-up energy generation.}} For traditional grid architectures where generation is centralized and under the complete control of the utility, it is common to monitor power quality only to the substation level, because~it can be assumed that the power quality experienced at the substation is a reasonably accurate proxy for the power quality experienced by the 1000 or so end-users serviced by the substation. Furthermore, if~an end-user experiences a power quality problem not experienced by the substation, then it is most likely due to equipment or electrical issues local to that end-user and not a grid-level~problem.

These assumptions might not hold in grids with distributed, intermittent generation by end-users, such as is the case with roof top solar. In~these cases, the~unpredictable nature of generation can lead to local power quality problems that are not experienced by the substation, and~that are not due to any single individual, but~rather the bottom-up power generation architecture of the grid. Solar~panels connected to low voltage networks will result in over-voltages, the~switching frequency of the converters in wind turbines causes high-frequency signals flowing into the grid, and~harmonics are generated by EV chargers.~\cite{zavoda_power_2018}
Nakafuji notes that Hawaiian Electric is contending with PV penetrations in excess of 60\% on certain distribution circuits, for~which traditional rules of thumb for the design of protection and distribution systems might not hold~\cite{nakafuji_back--basics_2011}.

\highlight{{\em 3. Consumer electronics require high quality power.}} The rise of consumer electronics has raised the bar for what constitutes ``adequate'' power quality. Only a few decades ago, computers were a rare presence outside of labs and large institutions. Today, computers are everywhere: embedded in phones, washers, refrigerators, thermostats, and~so forth. These electronic devices not only have higher power quality requirements, but~some of them actually introduce power quality problems in the form of harmonic distortion. Poor power quality can result in electronic devices failing unpredictably, and/or decreasing their lifespan. Reduced input voltage can cause excessive power supply heat dissipation, resulting in reduced mean time between failures (MTBF). In~addition, rectifiers and DC-to-DC converter switching transistors draw high-peak currents, which raise their junction temperatures. These temperature excursions take a toll on semiconductor longevity. High input voltage can also puncture a power supply's rectifier and switching transistor junctions, causing MTBF reduction and eventual breakdown. High-voltage transients lasting microseconds can permanently wreck the power supply and its electronic equipment load.~\cite{dedad_when_2008}.

\highlight{{\em 4. Power quality data should be publicly available.}} Electrical utilities are not required to be totally transparent about the quality of power they provide to consumers. For~example, in~Hawaii, utilities are only required to make a ``best effort'' to provide non-harmful voltage and frequency values, and~are only required to publicly report about power outages of more than 3 min. There is no requirement for utilities to report potentially harmful levels of voltage, frequency, THD, or~transients as long as they do not lead to outages. Indeed, in~places like Hawaii, there is not yet even infrastructure in place to enable utilities to collect that information, even if they were asked to report on it. Providing a high quality, third party, publicly available source of power quality data not only benefits the public, it makes it much easier to perform research on ways to improve grid stability in the presence of renewable energy~generation.

Better public understanding of power quality is important because it is such a significant economic problem. Across all business sectors, the~U.S. economy is estimated to be losing between \$104 and \$164 billion a year to outages and another \$15 to \$24 billion to power quality phenomena~\cite{elphick_summary_2015}. India lost more than \$9.6 billion in 2008 due to power quality problems, and~Europe is estimated to be losing \$150 billion per year~\cite{laskar_power_2012}.

Providing power quality data to address the above issues requires solving several difficult technical problems. First, monitoring below the substation level increases the number of required monitoring devices by two to three orders of magnitude. This has significant implications for the cost of monitoring in terms of the devices, and~the sheer amount of low-level power quality data that will be produced. Second, this low-level data must be processed in a manner that does not require inordinate amounts of processing, network bandwidth, or~storage. Finally, collecting the data is not useful if it cannot be converted into actionable information in a reasonable amount of time. This paper describes results to date of the Open Power Quality project, which attempts to address these~problems.

\section{Methods}

The central aim of the OPQ Project is to provide a scalable source of actionable power quality data about electrical grids, particularly those with high levels of distributed, intermittent power~generation.

Our research methodology consists of three (potentially interleaved) phases. First, we specify, design, and~implement novel hardware and software for collecting, transmitting, and~analyzing four essential measures of power quality data: voltage, frequency, THD, and~transients. Second,~we~verify that our implementation matches our specifications through laboratory experiments. Third, we~validate our implementation by testing a variety of hypotheses through a real-world deployment. In~this paper, we report on results from a pilot deployment at the University of Hawaii over the course of three~months.

Our central hypothesis is that we can accomplish this through the design and implementation of a low-cost sensor network for power quality using custom power quality monitors with real-time, two-way communication to a set of cloud-based~services.

We will test this central hypothesis through six subhypotheses: (1) OPQ provides valid and reliable collection of power quality data (Section \ref{hyp:01}), (2) OPQ's triggering system provides advantages with respect to bandwidth and computation (Section \ref{hyp:02}), (3) OPQ enables subthreshold event detection based upon temporal locality (Section \ref{sec:subthreshold-events}), (4) the OPQ information architecture provides a means to produce actionable insights (Section \ref{hyp:04}), (5) the OPQ information architecture provides predictable upper bounds on storage resources (Section \ref{hyp:05}), and~(6) OPQ provides useful adaptive optimization capabilities (Section \ref{sec:adaptive-optimization}).

The remainder of this section presents the overall architecture of OPQ, along with details on the design and implementation of each of the principal architectural components. We will also discuss our verification activities for each component. Our validation and hypothesis testing activities will be described in Section~\ref{sec:pilot-study}.

\subsection{OPQ~Architecture}

The OPQ system architecture consists of four major open source hardware and software components that provide end-to-end support for the capture, triggering, analysis, and~reporting of consumer level local and global PQ events. OPQ Box is a hardware device that detects the electrical waveform from a standard residential outlet and communicates both low and high fidelity representations of the waveform to other OPQ system components either at predefined intervals or upon request. OPQ Makai monitors incoming low fidelity data from OPQ Boxes, requests high fidelity data when necessary, and~stores the results in a MongoDB database. OPQ Mauka analyzes low level data and analyzes it to produce higher-level representations according to the information architecture described below, and~can tell OPQ Makai to request high fidelity data from one or more OPQ Boxes to facilitate analysis. OPQ View is a browser-based visualization platform for displaying the results for data capture and~analysis.



Figure~\ref{fig:architecture} illustrates how these components work together to take information from wall outlets (on the left side) to the display of analyses in a browser (on the right hand side). First, OPQ Boxes analyze power from wall outlets, and~send low fidelity measurements to OPQ Makai. OPQ Makai analyzes low fidelity measurements, and~requests high fidelity waveforms when desirable. Both measurements and waveforms are saved in a MongoDB database. OPQ Mauka analyzes low and high fidelity data, and~creates ``events'' to represent anomalies. OPQ View notifies users of events and allows them to drill down into low and high fidelity~data.

OPQ Makai, OPQ Mauka, and~OPQ View are all cloud-based software services that collectively form a single ``instance'' with respect to data transmission, storage, analysis, and~visualization. We~refer to this collection of software-side components as OPQ Cloud. Every OPQ Box connects to a single instance of an OPQ Cloud. It is possible to have multiple OPQ Cloud instances. For~example, a~company might install an OPQ Cloud instance behind their firewall along with OPQ Boxes to provide a private mechanism for collecting and analyzing power quality~data.

\begin{figure}[H]
\center \includegraphics[width=5in]{images/architecture/system-diagram.png}
\caption{High level system architecture of an Open Power Quality (OPQ) Sensor~Network.}
\label{fig:architecture}
\end{figure}


\subsection{OPQ~Box}
\label{sec:opq-box}

OPQ Box is a hardware device designed to provide inexpensive, extensible and accurate residential power quality measurements.
A block diagram of OPQ Box is shown in Figure~\ref{fig:opq:2}a.
A complete device is shown in Figure~\ref{fig:opq:2}b.

\begin{figure}[H]
\centering
\begin{subfigure}{.5\textwidth}
\centering
\includegraphics[width=0.9\linewidth]{images/opq-box/opqbox_diagram.png}
\caption{OPQ Box Block Diagram.
The power path is in red, signal path is in green and the digital IO is in black.}
\label{fig:opq:1:1}
\end{subfigure}%
\begin{subfigure}{.5\textwidth}
\centering
\includegraphics[width=0.7\linewidth]{images/opq-box/opqbox_photo.jpg}
\caption{OPQ Box in an ABS plastic~enclosure.}
\label{fig:opq:1:2}
\end{subfigure}
\caption{(\textbf{a}) OPQ Box block diagram and (\textbf{b}) production OPQ Box ready for~deployment.}
\label{fig:opq:2}
\end{figure}
\unskip

\subsubsection{OPQ Box~Hardware}\label{subsec:hardware}

The power system of the OPQ Box electrically isolates most of the device from the AC mains power.
An isolated AC-DC converter generates $5V_{dc}$ from the mains $120V_{ac}$.
A total of \hl{5 V} is used to power the Raspberry Pi, the~equipment connected to the expansion port, the~3.3 V regulators and voltage reference, and~an isolated DC/DC converter.
A total of 3.3 V is used to power the isolated end of the isolation amplifier as well as the STM32F3 analog to digital converter/digital signal processor (ADC/DSP).
The~hot side of the isolation amplifier is powered from the isolated DC/DC converter.
This allows OPQ Box to function with a battery attached to the expansion port, so that it may retain data and continue to operate during a power~outage.

%%% We added a space between 5 and V. Please confirm. CONFIRMED.


A Raspberry Pi single board computer (SBC) is responsible for signal analysis and anomaly detection.
The Raspberry Pi model used in OPQ Box is the Pi Zero W equipped with 256 MB of main memory and a single core 1 GHz ARM11 CPU. It also contains an on-board 802.11n WIFI transceiver, which removes the need for an external WIFI~dongle.

\subsubsection{OPQ Box~Software}\label{subsec:software}

The software stack of the Raspberry Pi aims to deliver an accurate and precise power quality analysis framework despite the rather limited capabilities of the hardware.
A block diagram of the software stack is shown in Figure~\ref{fig:opq:3}.
Digital data is transferred from the DSP to the Raspberry Pi via Serial Peripheral Interface, with~the Pi acting as the master and the DSP as a slave device.
A hardware interrupt line is used to inform Pi software that the DSP is ready for the data transfer, and~a kernel driver provides management of the SPI bus.
Internally, the~OPQ driver maintains a ring buffer of 16 windows, each of which is 200 data samples in size.
Upon receiving the interruption for the DSP, the~CPU sets up the DMA transfer and the DMA engine transfers a 200 sample window into the kernel memory without CPU interaction.
This scheme requires the CPU to only service 60 interruptions a second, with~each interruption requiring on the order of 100 instructions, for~a CPU utilization of less than $1\%$ in normal operation.
Userland applications communicate with the kernel driver using a file descriptor, where every \highlight{$read$} %Is italics necessary here?
 system call yields 200 samples of raw waveform.
As a result, the~smallest window that a userland application may process is a single AC cycle of the grid~mains.



\begin{figure}[H]
\begin{center}
\includegraphics[width=0.9\textwidth]{images/opq-box/opqbox_software.png}
\end{center}
\caption{Block diagram of the OPQ Box 2 software~stack.}
\label{fig:opq:3}
\end{figure}

The userland component of the OPQ Box software is a multi-threaded extensible analysis framework called Triggering.
The reader thread is responsible for transferring and accumulating data from the kernel driver.
The smallest data buffer that the Triggering application processes at any given time is 10 grid cycles or 2k samples.
Once the cycles are transferred to the userland and timestamped, they are passed to the analysis thread for feature extraction, as~well as to the Raw Data Ring Buffer (RDRB).
Since internally all data is addressed using shared pointers, during~data duplication no copying is required.
RDRS is capable of buffering up to an hour of data before it is overwritten, resulting in the RDBS maximum size of~100 MB.

The analysis thread of the Triggering application performs feature extraction of the raw data windows of 2000 samples.
Four metrics are extracted from the data stream: (1) Fundamental frequency, (2) RMS Voltage, (3) Total Harmonic Distortion, and~(4) Transients. Let us briefly discuss how each of these are~computed.

\subsubsection{OPQ Box: Calculating Fundamental~Frequency}\label{subsec:fundamental-frequency}

The fundamental frequency is calculated by computing the zero crossings of the AC waveform.
In order to improve the accuracy of the frequency calculation one must first filter out as much noise as possible.
Since our sampling rate is quite high (12 kSps) and the fundamental frequency is quite low (60~Hz), it is very computationally expensive to perform this filtering in a single step.
Instead, filtering is accomplished via a set of two low pass finite impulse response (FIR) filters.
The first filter has a passband of 0--600 Hz allowing us to downsample the waveform to 1200 samples per second.
The next~filter has a passband of 0--100 Hz allowing for further removal of high frequency noise.
Finally, zero crossings are extracted and used for the frequency calculation.
The zero crossings themselves were calculated by using linear interpolation between two points that bracket the time~axis.

\subsubsection{OPQ Box: Calculating RMS~Voltage}\label{subsec:root-mean-square-voltage}

Root mean square voltage ($V_{rms}$) in electrical power is the equivalent value of DC voltage, which would dissipate the same power in the resistive load. $V_{rms}$ is a convenient measure for detecting voltage sags and swells, since they result in nominally higher and lower computed~value.

Similarly to the frequency calculation, OPQ Box uses a 10 cycle window for a single $V_{rms}$ calculation. Unlike the frequency calculation, the~input is not filtered a~priori.

\subsubsection{OPQ Box: Calculating~THD}\label{subsec:thd}

The OPQ Box calculates total harmonic distortion (THD) using the industry standard methodology.
It should be noted that in the power quality domain THD is expressed as a percentage as opposed to $\frac{dB}{\sqrt{Hz}}$ as used in other disciplines.
Operationally, OPQ Box computes THD for 10 cycles of the fundamental frequency.
First an FFT transforms the real voltage samples into its frequency components.
Next, the~square of the harmonic bins are accumulated and scaled by the magnitude of the fundamental~power.

\subsubsection{OPQ Box: Transient~Detection}\label{subsec:transient-detection}

OPQ Box transient detection is performed via filtering out of the fundamental frequency via an FIR high pass filter with a cutoff frequency of 400 Hz and searching for a maximum value in the~remainder.

It should be noted that this transient detection method is susceptible to THD fluctuations, since any harmonic above 400 Hz will remain in the filtered waveform.
However, since the THD information is transmitted along with the transient detection metric, they can be correlated in downstream transient~detection.

\subsubsection{OPQ Box: Network~Communication}\label{subsec:network-communication}

The computed fundamental frequency and $V_{rms}$ are transmitted to the Makai service for aggregation.
Data transmission is handled using ZeroMq software stack with Curve25519 elliptic curve encryption.
Each device holds a unique private and public key, as~well as the servers' public key, allowing both the Makai service and the OPQ Box to verify its peer.
Internally, metrics transmission uses ZeroMq's PUB/SUB protocol.
This protocol is a publish/subscribe topology, with~each message containing the topic and a payload.
Additionally, ZeroMq allows for multiple sub peers with subscriptions forwarded to the publisher automatically via a side channel.
This allows for the aggregation service to be spread across multiple nodes, with~minimal network~overhead.

If the aggregation service determines that an anomaly has occurred, it is able to request raw waveform from the OPQ Box RDRB via a separate ZeroMq pub sub channel.
If the RDRB buffer contains data for the requested temporal range, OPQ Box transmits the available data to the aggregation service via a push pull ZeroMq channel.
Protobuf message serialization is used to encode messages across the OPQ~ecosystem.

In order to make a distributed measurement, all of the OPQ Boxes on the OPQ network need to maintain an accurate time reference.
Time synchronization across multiple OPQ Boxes is accomplished using the Network Time Protocol.
The expansion port of the OPQ Box supports a GPS receiver.
However, since GPS receivers require line of sight to the sky, it was not used for deployment.
NTP performance has been verified against GPS resulting in time error of 8 $\pm$ 5 ms, which is typical for NTP running over the Internet with a nearby NTP~server.

\subsubsection{OPQ Box: Manufacturing}

Currently, there is no mechanism for mass production of OPQ Boxes, but~all of the plans are available under an Open Source license, so interested organizations with some basic hardware engineering skills can build their own~boxes.

Complete specifications for the OPQ Box hardware, firmware, and~software are available~\cite{negrashov_opq_2020}. As~of the time of writing, a~single OPQ Box can be manufactured for approximately \$100 in parts. The~cost drops significantly with scale, for~example, 100 OPQ Boxes can be manufactured for a cost of approximately \$75 in parts per~device.



\subsection{OPQ~Makai}
\label{sec:opq-makai}

OPQ Box provides an inexpensive hardware device for collecting four important power quality measures with high fidelity, but~realizing its potential requires an innovative approach involving two-way communication between OPQ Boxes and the OPQ cloud-based services. To~see why, consider the IEEE 1159 standard for single location power quality monitoring~\cite{unruh_ieee_2018}. For~transient monitoring, IEEE 1159 suggests a sampling rate of at least 7680 samples/second, up~to 1 Megasample/second. This~implies that if the cloud service requires the high fidelity data from all OPQ Boxes, it would incur a very large bandwidth cost. At~20 Ksamples/second with 16bit samples, a~single OPQ Box will generate 300 Kb/s of network bandwidth. Several thousand devices would easily saturate a 1~GB network link. In~addition, collecting and recording all of the raw waveform data from residential power quality meters could lead to security and privacy~issues.

Network bandwidth saturation is a common problem for distributed sensor networks, and~a common solution is called ``self-triggering''. In~this approach, each monitoring device is configured with a threshold value for one or more measures of interest. When the threshold for a measure of interest is exceeded, then and only then is data sent over the network to cloud-based services for~analysis.

The problem with the self-triggering approach is that grid-wide power quality events do not affect the entire grid in the same way. For~example, due to the grid’s hierarchical structure, a~voltage sag on one sub-circuit can manifest as a sag of a different magnitude or even a swell on another~\cite{kahle_power_2015}. This may result in a situation where some of the monitoring devices will not consider a power quality anomaly as an event, because~it did not surpass the metric threshold, and~simply ignore it. From~an analysis perspective, however, it can be useful to get raw data from all of the affected devices, not just the ones that were affected to the point where the box was triggered. This additional information can be used to localize the disturbance, as~well as better evaluate its~impact.

Since sending all the data is infeasible, and~since the self-triggering approach can potentially miss important forms of information, OPQ implements a novel, hybrid centralized/decentralized data acquisition scheme, which involves two-way communication between the OPQ Boxes and a cloud service called OPQ Makai. In~this scheme, OPQ Boxes use local processing resources to feature extract the incoming waveforms while storing them locally for an hour. Each OPQ Box sends its feature data to OPQ Makai once a second, which we called the ``triggering stream''. Feature data is very small, on~the order of a few kilobytes, and~so this approach allows the sensor network to scale to thousands of devices with acceptable network bandwidth requirements. OPQ Makai processes the incoming triggering stream and looks for anomalies. If~an anomaly is present in only a single device, it is highly probable that the cause is local and not grid-level. On~the other hand, if~the triggering stream shows an anomaly temporally collocated across multiple devices, the~entire network or a subset of the network may be queried for raw waveform data for a temporal region which corresponds to the disturbance in the triggering~stream.

Our pilot study, discussed in Section~\ref{sec:pilot-study} will provide examples of the novel analysis capabilities made possible by OPQ Box and OPQ Makai communication. In~general, here are the main advantages of our hybrid centralized/decentralized approach over traditional self-triggering and the ``naive'' approach of sending all of the data:

\highlight{{\em Bandwidth usage is minimized.}} % Is italics necessary?
 Instead of sending the entirety of raw data, only extracted features are sent. This results in a tiny fraction of the bandwidth requirement when compared to raw waveforms. Furthermore, the~temporal window, which encompasses a single feature, can be adjusted in real time. Thus, as~soon as an anomalous behavior is observed in a subset of sensors, this window can be adjusted for a finer grained feature~extraction.

\highlight{{\em Effects of latency are minimized.}} In this case, ``latency'' refers to the time required for OPQ Makai to process the incoming feature stream and decide whether to request high fidelity data from one or more OPQ Boxes. Even at 1 M samples/second at 16 bits of resolution, the~memory requirement to store 5 min of raw waveform without compression are on the order of 512 MB, which is well within the realm of inexpensive single board computers such as Raspberry PI. With~compression specifically suited to the signal of interest, the~memory requirement can be reduced even further. In~the case of the OPQ sensor network, OPQ Makai has an hour to process feature data and request high fidelity data from OPQ Boxes. During~our pilot study, OPQ Makai always responded within a second or~two.

\highlight{{\em Cloud processing requirements are reduced.}} Since feature extraction is already performed at the device level, cloud computation requirements are reduced. With~the advent of the Internet of Things, the~computational capacity of edge devices is~increasing.

\highlight{{\em Subthreshold data acquisition can improve understanding of grid-local anomalies.}} OPQ Makai makes the decision to acquire raw waveform from OPQ Boxes. This allows analysis of data from devices that were only mildly affected or even not affected at all by the disturbance. This creates new possibilities for investigation of disturbance propagation across the sensed area, as~will be discussed in Section~\ref{sec:pilot-study}.

\highlight{{\em Temporal locality allows OPQ to provide improved insights into power quality anomalies over traditional triggering algorithms.}} By exploiting the idea of temporal locality, it is possible to ascertain the geographical extent of an anomaly with only coarse features. This allows for a simple robust algorithm, which may be deployed at the sink node for anomaly~detection.

\subsubsection*{OPQ Makai: Design}

OPQ Makai is a distributed extensible microservice framework responsible for receiving the triggering stream from the OPQ Boxes, locating anomalous temporal regions and requesting raw waveform for the anomalous time ranges. As~shown in Figure~\ref{fig:makai-design}, Makai consists of four major components: Acquisition Broker, Triggering Broker, Event Service and the Acquisition~Service.

\begin{figure}[H]
\center \includegraphics[width=3in]{images/makai/makai_main.pdf}
\caption{Makai component~design.}
\label{fig:makai-design}
\end{figure}

The Triggering Broker is perhaps the simplest component of the OPQ Makai system. The~triggering stream generated by the OPQ Boxes is encrypted to preserve user privacy. In~order to minimize the CPU time spent decrypting the data across multiple OPQ services, the~Triggering Broker decrypts the data and sends clear text measurements to other OPQ cloud services. The~Triggering Broker uses the ZeroMq subscribe socket to receive data from OPQ Boxes, and~sends it via a publish socket to any connected client. Each publish message is published to a topic, which corresponds to the ASCII representation of the originating OPQ Box ID. This allows services that utilize the Triggering Broker to select a subset of IDs to operate on. This is useful for load balancing the backend services, or~dividing the OPQ network into separate regions with no electrical connections between~them.

The Acquisition Broker manages the two-way communication between the OPQ Boxes and the rest of the cloud infrastructure. Unlike the triggering stream, which originates from the OPQ Box, two-way communication is always initiated by OPQ cloud services. Two way communication is realized via a command response interface, where the OPQ service initiates the communication by sending a clear text command to the Acquisition Broker, which then forwards it in encrypted form to the appropriate OPQ~Boxes.

The Acquisition Service resides between the Triggering and Acquisition Brokers. The~Acquisition Service is responsible for three tasks:
(1) Computation of statistics of the incoming triggering stream; (2) Hosting plugins for triggering stream analysis; and (3) Generating data event requests for OPQ Boxes. The~Acquisition Service accesses the triggering stream by connecting to the publish socket of the Triggering broker. Since the connection is managed through the ZeroMq publish-subscribe socket, several Acquisition Services can be connected to a single Triggering broker endpoint, each servicing a subset of OPQ Boxes by subscribing to only specific devices. The~Acquisition Service does not include any analysis capabilities by default. Instead, analysis is performed by shared library loadable plugins. These plugins can be loaded and unloaded at runtime, thus allowing live upgrading and testing of new analysis~methods.

The Event service is a microservice that stores raw data generated by OPQ Boxes in the MongoDB Database. On~initialization, the~Event service queries MongoDB database for the highest event number recorded so far, connects to the Acquisition Broker’s publish port, and~subscribes to all messages that start with the prefix ``data”. This allows the Event service to capture every response from OPQ Boxes generated from commands issued by the Acquisition service plugins. Once the Event service receives a data response with an identity containing an event token it has not seen before, it will increment the event number, and~store it in an internal key value~store.

\subsection{OPQ~Mauka}
\label{sec:opq-mauka}

The previous sections discussed the design of OPQ Box, a~custom hardware device for collecting four important measures of power quality, and~OPQ Makai, a~novel, hybrid centralized/decentralized data acquisition scheme, which involves two-way communication between the OPQ Boxes. As~a result of these two innovations, an~OPQ sensor network has the ability to collect and analyze high fidelity, low level data about power quality anomalies in a cost-effective, scalable~fashion.

There are remaining challenges to creating a useful power quality sensor network. First, the~data provided by OPQ Boxes is low-level, ``primitive'' data consisting of either features (i.e., frequency, voltage, THD, and~transients) or waveform data. However,~what we actually want is actionable insights into grid stability. For~example, we might want to know if a given anomalous data value is actually detrimental, or~we might want to be able to predict when a power quality event might occur in the future based upon the recognition of cyclical events in the historical~data.

A second challenge involves the potentially high volume of data that might accumulate in the cloud. Although~OPQ Box and OPQ Makai provide a scalable mechanism for communicating power quality data to the cloud services, it is still the case that, over~time, a~substantial amount of data could accumulate. One strategy is to simply store all of the data sent to the cloud forever. This means that data storage requirements will increase monotonically over time, making the sensor network more costly to maintain the longer it is in place. An~alternative strategy is to implement an algorithm to identify uninteresting (or no longer interesting) data and discard it. Ideally, such an algorithm would enable OPQ sensor network designers to calculate an upper bound on the total amount of cloud storage required as a function of the number of nodes (OPQ Boxes) in the~network.

OPQ Mauka addresses both of these issues. First, OPQ Mauka provides a multi-layered representation for structuring and processing DSN data. The~structure and processing at each layer is designed with the explicit goal of turning low-level data into actionable insights. Second, each layer in the framework implements a ``time-to-live'' (TTL) strategy for data within the level. This~strategy states that data must either progress upwards through the layers towards more abstract, useful representations within a fixed time window, or~else it can be discarded. The~TTL strategy is useful because when implemented, it allows DSN designers to make reasonable predictions of the upper bounds on data storage at each level of the framework adjusting for the number of sensors and power anomaly~probability.

TTL also makes possible a ``graceful degradation'' of system performance if those bounds turn out to be exceeded. For~example, consider a situation in which a power network enters a prolonged period of widespread power quality instability, where every OPQ Box is reporting continuous anomalous conditions with respect to voltage, frequency, THD, and~transients. This ``worst case scenario'' would lead to the potential situation of every OPQ Box trying to upload raw waveform data all the time. The~TTL system provides safeguards, in~that whatever low-level data has not been processed relatively quickly can be discarded. Thus, instead of the system potentially going down entirely, it could instead continue to operate at a reduced~capacity.

Figure~\ref{fig:mauka-data-model} illustrates the hierarchical data model for OPQ Mauka. This data model can be conceptualized as a multi-level hierarchy that adaptively optimizes data storage using a tiered TTL approach and provides a mechanism in which typed aggregated data is continually refined to the point of becoming actionable. The~data model also includes software components called ``actors'' that both move data upward through the levels and also apply optimizations downward through the levels. Actors are implemented through a plugin architecture, making it easy to experiment with the data model and improve it over~time.



The lowest layer of the hierarchy is the Instantaneous Measurements Layer (IML). The~IML contains “raw” data, in~other words, the~digitized waveform. IML data exists both on each OPQ Box (where it is available for up to the previous 60 min). It also exists in the cloud, in~the event that OPQ's triggering mechanism has caused a temporal interval of waveform data to be uploaded. IML data in the cloud has a TTL of 15 min: unless the waveform data is found to be useful by a cloud service within 15 min, it can be~discarded.


\begin{figure}[H]
\center \includegraphics[width=4in]{images/mauka/mauka-data-model.png}
\caption{Mauka data model~hierarchy.}
\label{fig:mauka-data-model}
\end{figure}

The second layer is the Aggregate Measurements Level (AML). The~AML stores power quality summary statistics sent either once per second or once per minute by each OPQ Box. These~summary statistics include the maximum, minimum, and~average values of voltage, frequency, THD, and~transient metrics over the time window. It is AML data that is used to initiate the triggering process of uploading IML data from the~cloud.

The third layer is the Detections Level (DL). This layer is responsible for processing the IML and AML data to produce a representation of an ``event'' with a precisely defined start and end time based upon examination of the waveform. As~will be discussed in Section~\ref{sec:subthreshold-events}, knowledge of the start and end time of a power quality anomaly allows investigation of how that anomaly might manifest itself elsewhere in the grid, even if this manifestation is not severe enough to produce over-threshold data~values.

The fourth layer is the Incident Level (IL). This layer starts to answer the question of whether the data is ``actionable'' by classifying the detected event according to various industry standards for power quality anomalies: IEEE 1159, ITIC, SEMI F47, and~so forth. For~example, an~event that falls into the ITIC ``prohibited'' region clearly indicates a power quality anomaly that requires further study and~intervention.

The fifth and final level is the Phenomena Level (PL). This layer contains the results of analyses that attempt to identify cyclic, and~thus predictable, power quality disturbances. It also contains analysis results regarding the similarity of various incidents, which can help uncover causal factors. Finally, it provides analyses for adaptive optimization of the OPQ Sensor Network. These optimizations can change the thresholds for individual boxes to either increase their sensitivity or decrease their sensitivity over specific intervals of time. The~ultimate goal of adaptive optimization is to help the network learn to acquire all of the data useful for analyses, and~only the data useful for analyses. We~are still in the early stages of exploring the potential of adaptive optimization in OPQ~networks.

\subsubsection*{OPQ Mauka: Actors}

The current capabilities of OPQ Mauka can be summarized in terms of its Actors, which are implemented as plugins. Nine of the most important Actors are described~below.

\highlight{{\em Makai Event Actor.}} % Please confirm if this is necessary.
 The Makai Event Actor is responsible for reading data newly created by OPQ Makai into OPQ Mauka. It performs feature extraction on the raw data stream and forwards those features (or the raw data) to subscribing Actor plugins. This allows OPQ Mauka to perform feature extraction once, and~allow use of those features by multiple~Actors.

\highlight{{\em Frequency Variation Actor.}} The Frequency Variation Actor classifies generic frequency sags, swells, and~interruptions as defined by the IEEE 1159 standard. Both duration and deviation from nominal are used to perform these classifications. Duration classifications include frequency deviations that last for less than 50 ns, between~50 ns to 1 ms, and~1 to 50 ms. Classifications for deviations from nominal are performed for values that are up to 40\% deviation from nominal. This Actor is able to classify frequency swells, frequency interruptions, and~frequency sags, leading to the creation of data at the Incident~Layer.

\highlight{{\em IEEE 1159 Voltage Actor.}} The IEEE 1159 Voltage Actor is used to classify voltage Incidents in accordance with the IEEE 1159 standard \hl{[29]}. In~general, this standard classifies voltage disturbances by duration and by magnitude. Voltage durations are classified from 0.5 to 30 cycles, 30 cycles to 3~s, 3 s to a minute, and~greater than 1 minute. Voltage deviations are classified in both the sag and swell directions as a percentage from nominal. Sags are generally classified between 10\% and 90\% of nominal while swells are generally classified from 110\% to 180\% of nominal. This Actor is capable of classifying voltage sags, swells, and~interruptions as defined by the standard, and~creating data at the Incident Layer if~appropriate.
%%% Should it be reference citation? YES
\highlight{{\em Box Optimization Actor.}} The Box Optimization Actor is responsible for sending and receiving typed messages to and from OPQ Boxes from OPQ Mauka. This Actor is capable of requesting the state of each OPQ Box (e.g., uptime, Measurement rate, security keys). It is also capable of adjusting the state of individual OPQ Boxes by changing things such as the Measurement and Trend rate or the sampling rate used by the~Box.

\highlight{{\em Future Phenomena Actor.}} The Future Phenomena Actor is responsible for creating Future or Predictive Phenomena. These Phenomena are used to predict Events and Incidents that may occur in the future. This plugin does not subscribe to any messages, but~instead utilizes timers to perform its work. By~default, this plugin runs every 10~min.

When a Future Phenomena Actor runs, it loads any active Periodic Phenomena found in the database. If~Periodic Phenomena are found, this Actor extrapolates possible Detection and Incident Layer data by first examining their timestamps and then extrapolating into the future using the mean period and the standard deviation. For~each timestamp in a Periodic Phenomena, the~mean period is added. If~the resulting timestamp is in the future, a~Future Phenomena is created using the time range of the future timestamp plus or minus the standard deviation of the Periodic~Phenomena.

When a Future Phenomena is created, timers are started in a separate thread signifying the start and end timestamps of the Future Phenomena. When the first timer runs, messages are sent to the Box Optimization Actor and the Threshold Optimization Actor instructing OPQ Box thresholds to be set lower and measurement rates to be set higher. This increases the chance of seeing an anomaly over the predicted time window. When the second timer runs, these values are reset to their default values. Thus, the~plugin increases fidelity and decreases thresholds over the period of a Future~Phenomena.

{\em ITIC Actor.} The ITIC Actor analyzes voltage to determine where it falls within the ITIC curve~\cite{thallam_power_2000}. The~ITIC curve is a power acceptability curve that plots time on the x-axis and voltage on the y-axis. The~purpose of the curve is to provide a tolerance envelope for single-phase 120V equipment. The~curve defines three regions. The~first region is ``No Interruption'' and generally includes all voltages with very short sustained durations. All events within this region have no noticeable effect on power equipment. The~second region, the~''No Damage Region'', occurs during voltage sags for extended periods of time. Power Events in this region may cause equipment interruptions, but~it will not damage the equipment. The~final region, the~''Prohibited'' region, is caused by sustained voltage swells and may cause damage to power equipment. This Actor determines if an event falls within the ``No Damage'' or ``Prohibited'' regions and if so, creates an Incident to record~this.

{\em SEMI F47 Actor.} The SEMI F47 Actor is similar to the ITIC Actor in that it plots voltage and duration against a power acceptability curve. In~this case, the~standard used is the SEMI F47 standard~\cite{djokic_sensitivity_2005}. Rather than using a point-in-polygon approach, this plugin reads the voltage features sequentially and uses a state machine to keep track of the current classification. This plugin only classifies values as a ``violation'' or as~''nominal''.

{\em Transient Actor.} The Transient Actor is responsible for classifying frequency transients in power waveforms. The~plugin subscribes to messages from a topic that contains a calibrated power waveform payload. The~Transient Actor is capable of classifying impulsive, arcing, oscillatory, and~periodic notching transients. A~decision tree is utilized to select the most likely transient type and then further analysis is used to perform the actual classification of transients. Dickens et al.~\cite{dickens_transient_2019} provide more details on the transient classification system used by this~Actor.

{\em Periodicity Actor.} The Periodicity Actor is responsible for detecting periodic signals in power data. This Actor does not subscribe to any messages, but~instead runs off of a configurable timer. The~Actor is set to run by default once an hour and every hour it scrapes the last 24 h worth of data and attempts to find periods in the Measurements over that~duration.

For each feature in the Measurement and Trend data (e.g., frequency, voltage, and~THD), the~Periodicity Actor first removes the DC offset from the data by subtracting the mean. Next, the~Actor filters the signal using a fourth order high-pass filter to filter out noise. The~Actor then performs autocorrelation on the signal followed by finding the peaks of the autocorrelation. The~mean distance between the peaks of the autocorrelation provides the period of the~signal.

The Periodicity Actor only classifies data as periodic if at least three peaks were found and the standard deviation of the period is less than 600 s (10 min). Once a positive identification has been made, peak detection is performed on the original signal. Once the plugin has the timestamps and deviations from nominal of the periodic signal of interest, the~plugin can group Measurements, Trends, Detection Layer Events, and~Incidents that were created during the periodic signals together as part of the Periodic~Phenomena.

\subsection{OPQ~View}
\label{sec:opq-view}

The final component of the OPQ Sensor Network system architecture is called OPQ View. It is a web application, implemented using the Meteor application framework, which provides a variety of visualization and query services. An~example of the OPQ View home page is provided in Figure~\ref{fig:opq-view-home}.

\begin{figure}[H]
\center \includegraphics[width=5in]{images/view/homepage.png}
\caption{OPQ View: home~page.}
\label{fig:opq-view-home}
\end{figure}

The home page provides three components: a ``System Stats'' window, which indicates the number of data elements currently at each level of the OPQ Mauka data hierarchy, a~''System Health'' window that indicates whether the Cloud services appear to be running correctly and the status of communication with all known OPQ Boxes on the sensor network, and~a ``Box Trends'' visualization that provides the ability to quickly see trends in the four basic measures (Voltage, Frequency, THD, and~Transients) over~time.

Figure~\ref{fig:opq-view-box-map} shows a map-based view of the sensor network. This figure shows the location of 11 OPQ Boxes on the University of Hawaii campus at one point during Fall of 2019. Depending on the zoom level of the interface, some boxes are collapsed into disks with a number indicating the number of boxes at that location. Zooming in reveals more information about the box including near-real time values for voltage and frequency, as~is shown for the box in the upper right corner of the~figure.

\begin{figure}[H]
\center \includegraphics[width=5in]{images/view/boxmap-2.png}
\caption{OPQ View: Box~map.}
\label{fig:opq-view-box-map}
\end{figure}

Figure~\ref{fig:opq-view-incident-summary} shows a visual display of a single Incident. This view includes a map-based location of the box that was involved in the Incident, the~start and end time and duration of the Incident, the~classification(s) of the anomalous data, and~the waveform associated with the Incident (when applicable). If~additional analysis is desired, the~raw data can be downloaded in CSV~format.

\begin{figure}[H]
\center \includegraphics[width=5in]{images/view/incident-summary.png}
\caption{OPQ View: Incident~summary.}
\label{fig:opq-view-incident-summary}
\end{figure}

The above images provide a sense for how OPQ View helps users to understand, monitor, and~assess an OPQ Sensor Network and the underlying power quality of the grid it is attached to. There are several features in OPQ View for user and box management; for details, please see the OPQ View~documentation.

While OPQ View provides a variety of useful visualization and analysis features, users wishing to understand the power quality of their grid are not restricted to its current feature set. The~OPQ database schema is public, and~if a new analysis is desired, it is straightforward to query the database directly for the data of interest and build a new analysis off of~it.















\section{Related~Work}
\label{sec:related-work}

This section explains how OPQ fits into current industry solutions as well as academic research on power quality monitoring and analysis. For~the purposes of this review, we exclude utility-side power quality monitoring and analysis~systems.

\subsection{Power Quality~Hardware}
\label{sec:commercial-pq-devices}

There exists a very wide variety of power quality hardware devices, including those made by Fluke~\cite{fluke_fluke_2020}, Dranetz~\cite{dranetz_dranetz_2020}, Elspec~\cite{elspec_elspec_2020}, PowerSide (formerly Power Standards Lab) \cite{powerside_powerside_2020}, ACR Systems~\cite{acr_acr_2020}, and~OpenZMeter~\cite{viciana_openzmeter_2018}.

All of the above devices contrast with the OPQ Box in similar ways. First, all of them collect a wider variety of power quality measures than OPQ Box, and~most have been certified according to one or more industry standards. Except~for OpenZMeter, they are generally designed to support industrial applications, where the goal is to ensure that the power being supplied to a building or plant is of adequate quality, and/or that the machinery in the plant is not degrading power quality. Apart from the PowerWatch monitor, all of them are attached to electrical mains using current transformers. Finally, all of them are designed for “stand alone” operation: each device can independently gather and assess power~data.

While the OPQ Box has much more limited functionality, it is designed to be manufactured for approximately \$75, which is 10 to 100 times less expensive than most commercial devices, and~similar is cost to OpenZMeter. The~most important distinguishing feature of OPQ Box is that it is designed to grid-level, not single point monitoring, and~thus incorporates features (such as two-way communication with the cloud, and~subthreshold triggering) that are not present in devices intended for ``stand alone''~capabilities.

\subsection{Power Quality~Software}
\label{sec:commercial-pq-software}

PQView~\cite{electrotek_concepts_pqview_2020}, PQSCADA Sapphire~\cite{elspec_ltd_pqscada_2016}, PQDIF~\cite{sabin_ieee_2020}, and~Grid Protection Alliance~\cite{grid_protection_alliance_grid_2020} are examples of software and/or software standards for manipulating power quality~data.

The differences between the way OPQ and the above systems store and manipulate power quality data arise from fundamentally different architectural assumptions and the historical background of the technology. PQView and PQSCADA Sapphire are designed to operate in a technology environment consisting of a large number of installed, “stand alone” power quality monitors built by different vendors. Their goal is to aggregate the data collected by these devices, and~in order to do so, they depend upon the PQDif standard as a way to obtain power quality data independent of the vendor and device generating it. This results in a kind of “store and forward” process: power quality data is captured and stored on the device, and~then periodically bundled into a PQDif file and sent to the database~software.

OPQ Cloud, on~the other hand, is designed only to support the capabilities of OPQ Boxes. OPQ Boxes, furthermore, have no “stand-alone” capability; they maintain continuous connection to the Internet and upload power quality data to cloud-based services as needed. This means that OPQ implements a very different approach to representing and transmitting data than PQDif. For~details on the representation, see the OPQ Data Model, and~for details on communication, see the OPQ Protobuf~protocol.

\subsection{Research~Systems}

Di Manno et al.~\cite{di_manno_user_2015} describes a PQ monitoring system called PiKu. Unlike OPQ, PiKu is designed as a hardware device for sensing power quality that is directly integrated into a PC. Systems with similar architectures include TRANSIENTMETER, described in Da Ponte et al.~\cite{daponte_transientmeter:_2000}, BK-ELCOM, described in Bilik et al.~\cite{bilik_modular_2007}, and~a system described in Xu et al.~\cite{xu_distributed_2012}.

There are also research projects based upon leveraging existing monitoring infrastructure. \mbox{Suslov et al.~\cite{suslov_distributed_2014}} describe a distributed power quality monitoring system based upon existing phasor measurement units installed by utilities. Sayied et al.~\cite{sayied_power_2013} describe a system designed using existing smart meters. Kucuk et al.~\cite{kucuk_extensible_2010} describe a similar system for the Turkish National Grid using utility grid monitoring~infrastructure.

Mohsenian-Rad~et~al.~\cite{mohsenian-rad_distribution_2018} designed the $\mu$PMU (phase measurement unit) system, which provides distributed power quality measurements over power grid distribution systems. The~$\mu$PMUs in conjunction with their backend software provide two types of analytics. Descriptive analytics provide information about the types and classifications of power quality issues that are observed within the power distribution grid. Predictive analytics are used to predict future power quality issues. The~authors describe their system as providing the ground for enabling future prescriptive analytics, which is the idea of self-tuning the DSN to prepare for future power quality problems by using a combination of descriptive and predictive analytics. In~contrast, as~we will discuss in Section~\ref{sec:adaptive-optimization}, the~OPQ sensor network already supports self-tuning through adaptive~optimization.

One research system very similar in spirit to OPQ is FNET~\cite{liu_distribution_2017}. Like OPQ, the~FNET system consists of custom hardware that monitors the electrical signal from a wall outlet, and~uploads data to the cloud for further processing. Unlike OPQ, FNET is designed for monitoring of frequency disturbances, how they propagate across wide area (i.e., nation-wide) utility grids, and, if~possible, where the frequency disturbance originated. This means that FNET devices must be synchronized using GPS, and~that the data collected consists of frequency and voltage angle. OPQ is designed for more “local” grid analysis, and~we are not interested in propagation. As~a result, OPQ Boxes are synchronized using NTP rather than GPS, which reduces cost and simplifies installation (OPQ Boxes do not need line of site to a GPS satellite). Finally, FNET hardware appears to support only “one way” communication from device to the cloud, while OPQ Boxes support ``two way'' communication (from box to cloud, and~from cloud to box).

\section{Results and~Discussion}
\label{sec:pilot-study}

To evaluate the capabilities of the OPQ Sensor Network, we deployed 16 OPQ Boxes at the University of Hawaii Manoa campus over the course of three months in the Fall of 2019. The~University of Hawaii campus is an isolated microgrid connected to the Oahu powergrid only via a single 46 kV feeder. The~UH Campus also has commercial electrical meters (a mixture of GE PQMII and GE EPM 7000) deployed across various levels of the power delivery infrastructure. While the primary purpose of these meters is to monitor power consumption, they do include power quality monitoring capabilities. Data from these meters were used as ground truth for validation studies of the OPQ sensor~network.

Our pilot study is significant because much of the literature on power quality assessment relies on models, not actual installations~\cite{anurangi_effects_2017,bayindir_effects_2016,farhoodnea_power_2012,shafiullah_experimental_2014}. In~other cases, data was collected from only one location or for a very short time span~\cite{kucuk_assessment_2013,viciana_openzmeter_2018}.

\subsection{Descriptive~Statistics}

The pilot study started on 7 October 2019 and ended on 4 February 2020. We deployed 16 OPQ Boxes across campus. As~noted above, each OPQ Box collects 200 measurements per grid cycle, for~a total of approximately 1B raw measurements per day per box. Values for the maximum and minimum voltage, frequency, and~THD, along with the presence or absence of transients, is sent once a second to the cloud by each box. Thus, each box sends approximately 86,400 measures per day to the cloud at a minimum. Over~the course of the pilot study, a~total of approximately 116M aggregate measures were sent by all of the OPQ~Boxes.

Figure~\ref{fig:statistics} provides two additional sets of summary statistics regarding the pilot study. Figure 9a shows the number of anomalous events (i.e., where threshold values for frequency, voltage, THD, or~transients were exceeded) along with the number of OPQ Boxes that experienced the power quality anomaly during that same time period. So, for~example, 170,925 power quality anomalies were experienced by only one of the 16 OPQ boxes, and~463 anomalous power quality measurements were experienced by all 16 OPQ~Boxes.

\begin{figure}[H]
\centering
\begin{subfigure}{.5\textwidth}
\begin{tabularx}{\textwidth}{XXXX|}
\toprule
\textbf{Events} & \textbf{Boxes} & \textbf{Events} & \textbf{Boxes} \\
\midrule
170,925 & 1 & 203 & 9 \\
1654 & 2 & 160 & 10 \\
1109 & 3 & 162 & 11 \\
853 & 4 & 130 & 12 \\
593 & 5 & 169 & 13 \\
416 & 6 & 210 & 14 \\
354 & 7 & 477 & 15 \\
246 & 8 & 463 & 16 \\
& & & \\
\bottomrule
\end{tabularx}
\caption{Summary statistics: Events}
\end{subfigure}%
\begin{subfigure}{.5\textwidth}
\begin{tabularx}{\textwidth}{lX}
\toprule
\textbf{Incident} & \textbf{Total} \\
\midrule
Frequency Swell & 291,235 \\
Frequency Sag & 244,286 \\
Excessive THD & 21,395 \\
Voltage Sag & 620 \\
ITIC (No Damage) & 93 \\
SEMI F47 (Violation) & 24 \\
Voltage Interruption & 16 \\
Frequency Interruption & 14 \\
Voltage Swell & 8 \\
\bottomrule
\end{tabularx}
\caption{Summary statistics: Incidents}
\end{subfigure}
\caption{Summary statistics of (\textbf{a}) boxes involved in events and (\textbf{b}) incident type~occurrences.}
\label{fig:statistics}
\end{figure}
\unskip

\subsection{OPQ Provides Valid and Reliable Collection of Power Quality~Data}
\label{hyp:01}

The first goal of the pilot study was to assess how well an OPQ sensor network is able to collect basic power quality data. To~do this, we compared data on voltage and frequency collected by OPQ Boxes with data on voltage and frequency collected by the existing UH power monitors. The~results are shown in Figure~\ref{fig:opqbox-f-v-validation}.

\begin{figure}[H]
\centering
\begin{subfigure}{.5\textwidth}
\centering
\includegraphics[width=0.9\linewidth]{images/pilot/opqbox-frequency-validation.png}
\caption{Frequency~differences.}
\label{fig:opqbox-validation-1}
\end{subfigure}%
\begin{subfigure}{.5\textwidth}
\centering
\includegraphics[width=0.9\linewidth]{images/pilot/opqbox-voltage-validation.png}
\caption{Voltage~differences}
\label{fig:opqbox-validation-2}
\end{subfigure}
\caption{Validation of (\textbf{a}) frequency and (\textbf{b}) voltage.}
\label{fig:opqbox-f-v-validation}
\end{figure}

As the charts illustrate, there is very close correspondence between the building meters and the OPQ Boxes. For~frequency differences, the~value of $\sigma$ is 0.0079 Hz, and~for voltage, $\sigma$ is 0.1703 V. (Note that typical thresholds for frequency and voltage PQ anomalies is 1 Hz and 6 V, so OPQ Box values appear accurate enough for their intended purpose.)

Validating THD and transient data, the~other two basic power quality measures collected by OPQ Boxes, was more~challenging.

Figure~\ref{fig:opqbox-thd-validation} shows the results of comparing values of THD collected by OPQ Boxes and building~meters.

\begin{figure}[H]
\centering
\includegraphics[width=0.8\linewidth]{images/pilot/opqbox-thd-validation.png}
\caption{Total harmonic distortion (THD)~differences.}
\label{fig:opqbox-thd-validation}
\end{figure}

During the hours of 6 p.m. and 6 a.m., OPQBox and the building level meter displayed a high level of agreement, as shown in the blue histogram in Figure~\ref{fig:opqbox-thd-validation}.
On the other hand, during~the hours of 6 p.m. and 6 a.m., there was a static disparity of 0.13\% between the two meters as portrayed in the red histogram.
This is likely attributed to the meter location in the power transmission hierarchy.
While~the OPQ Box is plugged the $120V_{ac}$ line, the~building meter is monitoring the $480V_{ac}$ three phase line.
An additional active conditioning system installed along side the transformer responsible for compensating for reactive power in the building is the likely culprit in the~disparity.

Unfortunately, we were unable to perform validation of transient data collection, because~building meters did not provide us with access to that information.
However, synthetic tests performed in the lab against a calibration source showed that OPQ Box is able to measure transient magnitude with $\sigma=0.125$ V, significantly higher than the triggering threshold.
It should be noted that the internal transient metric provided by the OPQ Box is only used for event detection, while higher level analytics compute their own transient classification parameters.
As such, the~transient detection capabilities of the OPQ Box are more than sufficient for its role in the OPQ~ecosystem.

\subsection{OPQs Triggering System Provides Advantages with Respect to Bandwidth and~Computation}
\label{hyp:02}

The pilot study provided an opportunity to collect data on the resources required by an OPQ sensor network with respect to cloud-level network resources and server-level computation~overhead.

To assess network bandwidth utilization, we analyzed the data stream of frequency and voltage collected by OPQ Boxes, and~the resulting network bandwidth utilized by the OPQ Makai triggering algorithm. Over~the course of a typical day, OPQ Makai requested approximately 136 MB of data from the deployed OPQ Boxes. We then calculated how much data would be sent by a power quality meter using a more conventional triggering approach in which exceeding a threshold would automatically result in sending waveform data, and~found that, under~the same conditions, approximately 1025 MB would be sent to the cloud, or~eight times the network~bandwidth.

The computational cost of an OPQ sensor network is proportional to the amount of data acquired. Since the meter level analytics are at best only good enough for event detection, further analysis is required for event classification.
This implies that the OPQ Makai triggering algorithm is at least eight times more efficient than the self triggered counterpart, since eight times less data is sent by OPQ Boxes to the cloud for~analysis.

\subsection{OPQ Enables Subthreshold Event Detection Based on Temporal~Locality}
\label{sec:subthreshold-events}

One interesting capability enabled by two-way communication between OPQ Boxes and their cloud-based services is what we term {\em subthreshold event detection based on temporal locality}.
In a nutshell, when one OPQ Box determines that a power quality measure has exceeded a threshold, one of the actions of the sensor network is to request high fidelity waveform data from neighboring boxes for the same temporal period, regardless of whether those boxes are reporting over-threshold data.
In the event that any of those boxes actually do report over-threshold data, then the request ripples outward to the boxes neighboring that box, and~so forth.
This is accomplished by maintaining a model of the OPQ Box metrics for each device.
Once a single device passes a threshold, OPQ Makai will search the network for all other devices, which exhibit measurements that deviate significantly from the device model.
If more devices are located, data from all the affected devices will be captured.
Otherwise, the~event is deemed local and~ignored.

Sub-threshold triggering allows for clustering of OPQ Boxes based on their electrical distance without a priori knowledge of the power grid topology.
A distance metric comprised of the magnitude of the $V_{rms}$ disturbance between every pair of OPQ Boxes was used as the clustering parameter.
This~distance metric describes how much local disturbances observed by a device affect every other device on the network.
Hierarchical clustering employed on the aforementioned distance metric results in the hierarchy shown in Figure~\ref{fig:localization-clustering}a.
These clusters closely corresponded to the topology of the power grid.
For example, the two top-level clusters correspond to the two 12 kV feeders, which power the university campus.
As such, disturbances that originate on one feeder, have a minimal impact on the other.
On the other hand, devices 1009 and 1008 are located in adjacent buildings, and~a common 480 V feeder, resulting in a large level of commonality in observed anomalies.
Sub-threshold triggering along with clustering in turn made it possible to perform limited localization of PQ incidents without prior knowledge of the grid layout.
For example, consider a $V_{rms}$ sag observed by the OPQ network shown in Figure~\ref{fig:localization-clustering}b.
This event had the largest impact on devices 1008 and 1009, which, as shown via hierarchical clustering, are closely coupled.
From this information it is possible to conclude that the event depicted in Figure~\ref{fig:localization-clustering}b originated from inside the UH power grid relatively close to devices 1008 and~1009.

\begin{figure}[H]

\centering
\begin{subfigure}{.5\textwidth}
\centering
\includegraphics[width=0.9\linewidth]{images/pilot/clustring.png}
\caption{Hierarchical clustering of OPQ~Boxes.}
\label{fig:clustering}
\end{subfigure}%
\begin{subfigure}{.5\textwidth}
\centering
\includegraphics[width=0.9\linewidth]{images/pilot/localisation.png}
\caption{Localization of a~disturbance.}
\label{fig:localisation}
\end{subfigure}
\caption{Hierarchical clustering of OPQ Boxes, and~localization of a~disturbance.}
\label{fig:localization-clustering}
\end{figure}
\unskip

\subsection{The OPQ Information Architecture Provides a Means to Produce Actionable~Insights}
\label{hyp:04}

Figure~\ref{fig:level-statistics} provides some descriptive statistics on the total number of entities created at each layer of the OPQ Information Architecture over the course of the pilot study. Note that the number of entities at a given layer at a given point in time is typically far less, due to the use of TTL to discard entities not promoted to higher~layers.


Figure~\ref{fig:level-statistics} shows that all layers of the hierarchy were actively used. In~addition, the~system was able to produce actionable insights by automatically detecting periodic phenomena for two boxes, and~then predicting future occurrences of power quality anomalies with over a 50\% success rate (in other words, the~current implementation of predictive phenomena leads to a significant false positive rate, predicting twice as many future power quality anomalies as actually occurred). Improving the predictive capabilities is a topic of future~research.


\begin{figure}[H]
\centering
\begin{tabularx}{.4\textwidth}{lX}
\toprule
\textbf{Level} & \textbf{Total} \\
\midrule
Phenomena & 10.4K \\
Incidents & 415K \\
Detections & 91.4K \\
AML (Trends) & 1.94M \\
AML (Measurements) & 116~M \\
IML & 100B \\
\bottomrule
\end{tabularx}
\caption{Entities created at each level of the OPQ Information Architecture during the pilot~study.}
\label{fig:level-statistics}
\end{figure}

\subsection{The OPQ Information Architecture Provides Predictable Upper Bounds on Storage~Resources}
\label{hyp:05}

Figure~\ref{fig:data-management-graph} shows a graph that illustrates storage requirements both with and without TTL-based data removal. As~the graph shows, over~the course of the three month period, we estimate that cloud-level storage requirements would have reached approximately 2.5 TB if all data was kept. However, due to TTL, the~amount of data held in the cloud at the conclusion of the case study was only 100~GB.

\begin{figure}[H]
\centering
\includegraphics[width=0.4\linewidth]{images/pilot/data-management-graph.png}
\caption{\hl{Storage} requirements with and without~time-to-live (TTL).} %%% It seems that the figure isn't complete, please modify. FIGURE IS COMPLETE.
\label{fig:data-management-graph}
\end{figure}

It is difficult to compute a strict upper bound on storage requirements for an OPQ network, because~the amount of storage does depend on the number of power quality anomalies experienced by the grid. That said, we did notice that storage requirements were rising much more slowly at the end of the pilot when utilizing TTL as compared to not utilizing~TTL.

The largest contribution to data storage is the raw power data. In~a system without TTL, raw data storage rapidly grows. The~bounds on raw storage can be calculated by taking into account the sampling rate and sample size for active OPQ~Boxes.

Given that OPQ Boxes were configured to sample at 12 kHz at 2 bytes per sample during the deployment, we can calculate the bounds on raw storage for 16 OPQ Boxes over three months without TTL as 3.1~TiB.

Given a network of 16 OPQ Boxes over the pilot deployment period with TTL, the~combined data storage from all levels reached an asymptotic limit of near 100 GB. The~high rate of detections made by OPQ utilizes most of the space. The~reason for this is that Events store associated IML data along with each detection for high-fidelity analysis. It is also possible to observe how the AML and IML levels level off asymptomatically over time. This is a result of TTL discarding not useful~data.

\subsection{OPQ Provides Useful Adaptive Optimization~Capabilities}
\label{sec:adaptive-optimization}

The top layer of OPQ's information architecture supports adaptive optimization: the ability to analyze previously collected data and use it to change the settings associated with data collection from OPQ~Boxes.

Two areas where adaptive optimizations were utilized are identifying periodic phenomena and predictions of future phenomena. Periodic phenomena are PQ related signals that occur at regular intervals. One example of periodic phenomena is a cyclic voltage sag that was observed at one of our sensors with a period of about 34 min during our pilot study (Figure~\ref{fig:periodic-voltage-sags}). This periodic phenomena not only provided a classification of an interesting pattern, but~also helped optimize the OPQ system for the capture of future phenomena. Several system wide optimizations are~utilized.

\begin{figure}[H]
\centering
\includegraphics[width=0.90\linewidth]{images/pilot/periodic-voltage-sags.png}
\caption{Detection of periodic voltage~sags.}
\label{fig:periodic-voltage-sags}
\end{figure}

First, classified periodic phenomena are used to generate future phenomena. Future phenomena are a prediction of future signals of interest. When a prediction is made, the~system will automatically increase sampling fidelity for the predicted sensor during the time window of the prediction. In~particular, the~measurement rate is increased from one measurement per second to six measurements per second, providing higher fidelity data. Detection thresholds are also decreased, making it more likely that the predicted signal will be observed even if it is of low~magnitude.

Second, the~increase in detection sensitivity makes it more likely that sub-threshold signals of interest will be classified. For~example, the~increased detection sensitivity allowed us to observe and classify a periodic voltage sag that would not have been detected by our normal classifier due to the small magnitude of the~sag.

This creates a positive feedback loop whereas periodic phenomena become more accurate, future phenomena become more accurate. As~future phenomena become more accurate, periodic phenomena become more~accurate.






\section{Conclusions and Future~Directions}
\label{sec:conclusions}

This project has produced both hardware and software for power quality monitoring with a variety of innovations. Our OPQ Boxes collect frequency, voltage, THD, and~transients with high fidelity, and~at a cost that is generally 10$\times$ to 100$\times$ cheaper than current commercial offerings (though these commercial offerings offer a variety of features not available from OPQ Boxes, so the appropriate choice depends upon the needs of the user).

Our sensor network provides real-time, two-way communication between the sensor nodes (OPQ Boxes) and their cloud services, providing several innovative features for power quality monitoring. The~OPQ triggering system exploits temporal locality to request high fidelity data from neighboring boxes when one exceeds a threshold, enabling our network to detect and analyze data that would have been unreported by a naive, threshold-based approach. The~OPQ Information Architecture enables Actors at the Phenomena Layer to control sensitivity settings of individual OPQ Boxes in order to improve data collection when power quality anomalies have been~predicted.

Our sensor network is designed to efficiently use network and storage resources. Two way communication enables OPQ Boxes to send low fidelity summary statistics on power quality at one second intervals, which can be used by cloud-based services to decide whether to request high fidelity data. Our TTL mechanism implements a “use it or lose it” approach to cloud-based data, which in our pilot study reduced cloud storage requirements by over~ten-fold.

Our pilot study has provided evidence that OPQ sensor networks can provide useful new support for understanding grid stability, particularly in grids with distributed, intermittent renewables. This~in part because OPQ Boxes are suitable for installation at the residential level: installation requires only an available wall socket and WiFi connection, no electrician or GPS line-of-sight required. Finally, the~low cost of OPQ Boxes means mass deployment across a neighborhood is not financially~infeasible.

One future goal is to partner with an organization that can enable OPQ Boxes to be produced at volume. We made the OPQ Boxes by hand for the pilot study, and~although we have had numerous requests for OPQ Boxes, we do not possess manufacturing~capacity.

With manufacturing capacity will come the ability to deploy OPQ sensor networks at a higher scale. We believe that a single sensor network can easily scale to hundreds of boxes before bandwidth and processing constraints become an issue. To~scale to many thousands of boxes, we believe a federated approach would work in which individual OPQ sensor networks communicate their findings to each other. Those networks could be designed around natural grid boundaries such as~substations.

A second goal is to explore ways to combine OPQ sensor networks with more traditional power quality monitoring tools and standards, such as PQDIF. For~example, it is possible that OPQ analyses could be improved with access to additional power data such as current, phase angle, and~so~forth.

A third goal is to build Actors that operate not only on power quality data but also environmental data such as wind, temperature, humidity, and~insolation. These additional data streams could yield valuable mechanisms for creating Predictive Phenomena for power quality anomalies associated with renewable energy sources. Ultimately, such understanding could lead to ways to significantly increase the amount of distributed renewable energy that can be incorporated into electrical~grids.

To conclude, a~recent paper by Mohsenian et al.~\cite{mohsenian-rad_distribution_2018} states, “the main challenge is to go beyond manual methods based on the intuition and heuristics of human experts...it is crucial to develop the machine intelligence needed to automate and scale up the analytics on billions of PMU measurements and terabytes of data on a daily basis and in real time.” We believe that the OPQ sensor network represents a small step along the path toward that~future.


%%%%%%%%%%%%%%%%%%%%%%%%%%%%%%%%%%%%%%%%%%
\authorcontributions{\hl{For} }%% research articles with several authors, a short paragraph specifying their individual contributions must be provided. The following statements should be used ``Conceptualization, X.X. and Y.Y.; methodology, X.X.; software, X.X.; validation, X.X., Y.Y. and Z.Z.; formal analysis, X.X.; investigation, X.X.; resources, X.X.; data curation, X.X.; writing--original draft preparation, X.X.; writing--review and editing, X.X.; visualization, X.X.; supervision, X.X.; project administration, X.X.; funding acquisition, Y.Y. All authors have read and agreed to the published version of the manuscript.'', please turn to the  \href{http://img.mdpi.org/data/contributor-role-instruction.pdf}{CRediT taxonomy} for the term explanation. Authorship must be limited to those who have contributed substantially to the work reported. SEE BELOW

\authorcontributions{
Conceptualization, Johnson, Christe, Negrashov; methodology, Johnson, Christe, Negrashov; software, Johnson, Christe, Negrashov; validation, Johnson, Christe, Negrashov; formal analysis, Johnson, Christe, Negrashov; investigation, Johnson, Christe, Negrashov; resources, Johnson; data curation, Christe, Negrashov; writing--original draft preparation, Johnson, Christe, Negrashov; writing--review and editing, Johnson, Christe, Negrashov; visualization, Johnson, Christe, Negrashov; supervision, Johnson; project administration, Johnson; funding acquisition, Johnson. All authors have read and agreed to the published version of the manuscript.
}

%%%%%%%%%%%%%%%%%%%%%%%%%%%%%%%%%%%%%%%%%%
\funding{\hl{Please} }%add: ``This research received no external funding'' or ``This research was funded by NAME OF FUNDER grant number XXX.'' and  and ``The APC was funded by XXX''. Check carefully that the details given are accurate and use the standard spelling of funding agency names at \url{https://search.crossref.org/funding}, any errors may affect your future funding.

\funding{This research was funded in part by a grant from the University of Hawaii President's Green Award program.}

%%%%%%%%%%%%%%%%%%%%%%%%%%%%%%%%%%%%%%%%%%
\conflictsofinterest{\hl{Declare} } %conflicts of interest or state ``The authors declare no conflict of interest.'' Authors must identify and declare any personal circumstances or interest that may be perceived as inappropriately influencing the representation or interpretation of reported research results. Any role of the funders in the design of the study; in the collection, analyses or interpretation of data; in the writing of the manuscript, or in the decision to publish the results must be declared in this section. If there is no role, please state ``The funders had no role in the design of the study; in the collection, analyses, or interpretation of data; in the writing of the manuscript, or in the decision to publish the results''.

\conflictsofinterest{The authors declare no conflict of interest.}


% Bibliography
\reftitle{References}
\begin{thebibliography}{999}
\providecommand{\natexlab}[1]{#1}

\bibitem[Trabish(2014)]{trabish_solar_2014}
Trabish, H.
\newblock Solar Installers flee {Hawaii} as Interconnection Queue Backs Up.
\newblock {\em Utility Dive}, {\hl{September 29, 2014}}.%% Please add day month. DONE

\bibitem[Anastasi \em{et~al.}(2009)Anastasi, Conti, Di~Francesco, and
Passarella]{anastasi_energy_2009}
Anastasi, G.; Conti, M.; Di~Francesco, M.; Passarella, A.
\newblock Energy conservation in wireless sensor networks: {A} survey.
\newblock {\em Ad~Hoc Networks} {\bf 2009}, {\em 7},~537--568, doi:10.1016/j.adhoc.2008.06.003.

\bibitem[Anurangi \em{et~al.}(2017)Anurangi, Rodrigo, and
Jayatunga]{anurangi_effects_2017}
Anurangi, R.O.; Rodrigo, A.S.; Jayatunga, U.
\newblock Effects of high levels of harmonic penetration in distribution
networks with photovoltaic inverters.
\newblock In {Proceedings of the } 2017 {IEEE} {International} {Conference} on IEEE Industrial and {Information} {Systems} ({ICIIS}), Jayapura, Indonesia \hl{November 2, 2017}; pp. 1--6, doi:10.1109/ICIINFS.2017.8300335.%% Please add conference location and date.

\bibitem[{Rita Pinto} \em{et~al.}(2016){Rita Pinto}, {Sílvio Mariano}, {Maria
Do Rosário Calado}, and {José Felippe de Souza}]{rita_pinto_impact_2016}
Pinto, R.; Mariano, S.; Calado, M.D.R.;  De Souza, J.F.
\newblock Impact of {Rural} {Grid}-{Connected} {Photovoltaic} {Generation}
{Systems} on {Power} {Quality}.
\newblock {\em Energies} {\bf 2016}, {\em 9},~739, doi:10.3390/en9090739.

\bibitem[Delgado \em{et~al.}(2011)Delgado, Santos, de~Almeida, and
Figueira]{delgado_solutions_2011}
Delgado, J.; Santos, B.; de~Almeida, A.T.; Figueira, A.
\newblock Solutions to mitigate power quality disturbances resulting from
integrating intermittent renewable energy in the grid of {Porto} {Santo}.
\newblock In {Proceedings of the } 11th {International} {Conference} on {Electrical} {Power} {Quality}
and {Utilisation}, \hl{October 17, 2011}; pp. 1--6,
\newblock ISSN 2150-6655,
doi:{\changeurlcolor{black}\href{https://doi.org/10.1109/EPQU.2011.6128870}{\detokenize{10.1109/EPQU.2011.6128870}}}.

\bibitem[Zavoda(2018)]{zavoda_power_2018}
Zavoda, F.
\newblock { Power {Quality} and {EMC} {Issues} with {Future} {Electricity}
{Networks}}. In {Proceedings of the } CIGRE, \hl{May 15, 2018}.

\bibitem[Nakafuji \em{et~al.}(2011)Nakafuji, Aukai, Dangelmaier, Reynolds,
Yoshimura, and {Ying Hu}]{nakafuji_back--basics_2011}
Nakafuji, D.; Aukai, T.; Dangelmaier, L.; Reynolds, C.; Yoshimura, J.; {
Hu, Y}.
\newblock “{Back}-to-basics”: {Operationalizing} data mining and visualization
techniques for utilities.
\newblock In {Proceedings of the }The 2011 International Joint Conference on Neural Networks, \hl{July 31, 2011}; pp. 3093--3098, doi:10.1109/IJCNN.2011.6033630.

\bibitem[Dedad(2008)]{dedad_when_2008}
Dedad, J.
\newblock When {Does} {Poor} {Power} {Quality} {Cause} {Electronics}
{Failures}?
\newblock {\em Electrical Construction and Maintenance Magazine}, {\hl{November 1, 2008}}.
\newblock Library Catalog.  Available online:  \url{www.ecmweb.com} (accessed on \hl{31 July 2020}).

\bibitem[Elphick \em{et~al.}(2015)Elphick, Ciufo, Smith, and
Parera]{elphick_summary_2015}
Elphick, S.; Ciufo, P.; Smith, V.; Parera, S.
\newblock Summary of the economic impacts of power quality on consumers.
\newblock  In {Proceedings of the }2015 {Australasian} {Universities} {Power} {Engineering}
{Conference}, \hl{September 27, 2015}.

\bibitem[Laskar(2012)]{laskar_power_2012}
Laskar, S.H.
\newblock Power quality issues and need of intelligent {PQ} monitoring in the
smart grid environment.
\newblock In~{Proceedings of the }2012 47th
{International} Universities {Power} {Engineering} {Conference} ({UPEC}), \hl{September 4, 2012}; pp.~1--6.

\bibitem[Negrashov(2020)]{negrashov_opq_2020}
Negrashov, S.
\newblock {OPQ} {Box} {Specifications} and {Design}, 2020.
\newblock Library Catalog.  Available online:  \url{github.com} (accessed on \hl{31 July 2020}).

\bibitem[Unruh(2018)]{unruh_ieee_2018}
Unruh, T.
\newblock \emph{IEEE P1159/D3: {Draft} {Recommended} {Practice} for
{Monitoring} {Electric} {Power} {Quality}}; IEEE, New York, NY \hl{February 1, 2018}. %% Please provide publisher and its location. DONE

\bibitem[Kahle(2015)]{kahle_power_2015}
Kahle, K.
\newblock Power {Converters} and {Power} {Quality}.
\newblock \emph{arXiv} \textbf{2015}, arXiv:1607.01556. doi:10.5170/CERN-2015-003.57.


\bibitem[Thallam and Heydt(2000)]{thallam_power_2000}
Thallam, R.; Heydt, G.
\newblock Power acceptability and voltage sag indices in the three phase sense.
\newblock In {Proceedings of the } 2000 {Power} {Engineering} {Society} {Summer} {Meeting},
 \hl{July 16, 2000}; Volume~2, pp. 905--910, doi:10.1109/PESS.2000.867482.

\bibitem[Djokic \em{et~al.}(2005)Djokic, Desmet, Vanalme, Milanovic, and
Stockman]{djokic_sensitivity_2005}
Djokic, S.; Desmet, J.; Vanalme, G.; Milanovic, J.; Stockman, K.
\newblock Sensitivity of personal computers to voltage sags and short
interruptions.
\newblock {\em IEEE Trans. Power Deliv.} {\bf 2005}, {\em
20},~375--383.
\newblock
doi:{\changeurlcolor{black}\href{https://doi.org/10.1109/TPWRD.2004.837828}{\detokenize{10.1109/TPWRD.2004.837828}}}.

\bibitem[Dickens \em{et~al.}(2019)Dickens, Christe, and
Johnson]{dickens_transient_2019}
Dickens, C.; Christe, A.; Johnson, P.
\newblock A {Transient} {Classification} {System} {Implementation} on an {Open}
{Source} {Distributed} {Power} {Quality} {Network}.
\newblock  In Proceedings of the {Ninth} {International} {Conference} on {Smart}
{Grids}, {Green} {Communications} and {IT} {Energy}-aware {Technologies},
\hl{June 2, 2019}.

\bibitem[Fluke(2020)]{fluke_fluke_2020}
Fluke, I.
\newblock \emph{Fluke {Corporation}: {Fluke} {Electronics}, {Biomedical},
{Calibration} and {Networks}}; \hl{July 31, 2020}.

\bibitem[Dranetz(2020)]{dranetz_dranetz_2020}
Dranetz, I.
\newblock Dranetz {Power} {Monitoring}, 2020.
\newblock Library Catalog. Available online:  \url{ www.dranetz.com} (accessed on \hl{31 July 2020}).

\bibitem[Elspec(2020)]{elspec_elspec_2020}
Elspec, I.
\newblock Elspec---{Power} Quality Analyzers and Solutions, 2020.
\newblock Library Catalog. Available online:  \url{www.elspec-ltd.com} (accessed on \hl{31 July 2020}).

\bibitem[Powerside(2020)]{powerside_powerside_2020}
Powerside, I.
\newblock Powerside, {Inc}., 2020.
\newblock Library Catalog. Available online:  \url{Powerside.com}  (accessed on \hl{31 July 2020}).

\bibitem[ACR(2020)]{acr_acr_2020}
ACR, S.
\newblock {ACR} {Systems}, 2020.
\newblock Library Catalog. Available online:  \url{www.acrdatasolutions.com}  (accessed on \hl{31 July 2020}).

\bibitem[Viciana \em{et~al.}(2018)Viciana, Alcayde, Montoya, Baños,
Arrabal-Campos, Zapata-Sierra, and
Manzano-Agugliaro]{viciana_openzmeter_2018}
Viciana, E.; Alcayde, A.; Montoya, F.; Baños, R.; Arrabal-Campos, F.;
Zapata-Sierra, A.; Manzano-Agugliaro, F.
\newblock {OpenZmeter}: {An} {Efficient} {Low}-{Cost} {Energy} {Smart} {Meter}
and {Power} {Quality} {Analyzer}.
\newblock {\em Sustainability} {\bf 2018}, {\em 10}, 4038.
\newblock
doi:{\changeurlcolor{black}\href{https://doi.org/10.3390/su10114038}{\detokenize{10.3390/su10114038}}}.

\bibitem[Concepts(2020)]{electrotek_concepts_pqview_2020}
Concepts, E.
\newblock {PQView} 4 {\textbar} {Electrotek} {Concepts}, 2020.
\newblock Library Catalog. Available online:  \url{www.electrotek.com}  (accessed on \hl{31 July 2020}).

\bibitem[{Elspec, Ltd}(2016)]{elspec_ltd_pqscada_2016}
{Elspec, Ltd}.
\newblock {PQSCADA} {Sapphire}, 2016.
\newblock Library Catalog. Available online:  \url{www.elspec-ltd.com}  (accessed on \hl{31 July 2020}).

\bibitem[Sabin(2020)]{sabin_ieee_2020}
Sabin, D.
\newblock \emph{{IEEE} 1159.3 {PQDIF} {Task} {Force}}; \hl{July 31, 2020}.

\bibitem[Alliance(2020)]{grid_protection_alliance_grid_2020}
Alliance, G.P.
\newblock \emph{Grid {Protection} {Alliance}---{Home}}; \hl{July 31, 2020}.

\bibitem[Di~Manno \em{et~al.}(2015)Di~Manno, Varalone, Verde, De~Santis,
Di~Perna, and Salemme]{di_manno_user_2015}
Di~Manno, M.; Varalone, P.; Verde, P.; De~Santis, M.; Di~Perna, C.; Salemme, M.
\newblock User friendly smart distributed measurement system for monitoring and
assessing the electrical power quality.
\newblock In Proceedings of the 2015 {AEIT} {International} {Annual}
{Conference}, \hl{October 14, 2015}.

\bibitem[Daponte \em{et~al.}(2000)Daponte, Di~Penta, and
Mercurio]{daponte_transientmeter:_2000}
Daponte, P.; Di~Penta, M.; Mercurio, G.
\newblock {TRANSIENTMETER}: A distributed measurement system for power quality
monitoring.
\newblock In {Proceedings of the } Ninth {International} {Conference} on {Harmonics} and {Quality} of
{Power}, Orlando, Florida \hl{October 1, 2000}.

\bibitem[Bilik \em{et~al.}(2007)Bilik, Koval, and Hula]{bilik_modular_2007}
Bilik, P.; Koval, L.; Hula, J.
\newblock Modular system for distributed power quality monitoring.
\newblock In {Proceedings of the } 9th {International} {Conference} on {Electrical} {Power} {Quality}
and {Utilization}, \hl{2007}.

\bibitem[Xu \em{et~al.}(2012)Xu, Xu, Xi, and Zhang]{xu_distributed_2012}
Xu, W.; Xu, G.; Xi, Z.; Zhang, C.
\newblock Distributed power quality monitoring system based on {EtherCAT}.
\newblock In {Proceedings of the }2012 {China} {International} {Conference} on {Electricity}
{Distribution}, \hl{2012}.

\bibitem[Suslov \em{et~al.}(2014)Suslov, Solonina, and
Smirnov]{suslov_distributed_2014}
Suslov, K.; Solonina, N.; Smirnov, A.
\newblock Distributed power quality monitoring.
\newblock In {Proceedings of the } {IEEE} 16th {International} {Conference} on {Harmonics} and
{Quality} of {Power}, \hl{2014}.

\bibitem[Sayied \em{et~al.}(2013)Sayied, Spaulding, and
Akbary]{sayied_power_2013}
Sayied, O.; Spaulding, J.; Akbary, B.
\newblock \hl{Power quality metrics calculations for the smart grid}. %%% Please add more detailed information.

\bibitem[Kucuk \em{et~al.}(2010)Kucuk, Inan, Salor, Demirci, Akkaya, Buhan,
Boyrazoglu, Unsar, Altintas, Haliloglu, Cadirci, and
Ermis]{kucuk_extensible_2010}
Kucuk, D.; Inan, T.; Salor, O.; Demirci, T.; Akkaya, Y.; Buhan, S.; Boyrazoglu,
B.; Unsar, O.; Altintas, E.; Haliloglu, B.; et al.
\newblock An extensible database architecture for nationwide power quality
monitoring.
\newblock {\em Electr. Power Energy Syst.} {\bf 2010}, {\em 32},  559--570.

\bibitem[Mohsenian-Rad \em{et~al.}(2018)Mohsenian-Rad, Stewart, and
Cortez]{mohsenian-rad_distribution_2018}
Mohsenian-Rad, H.; Stewart, E.; Cortez, E.
\newblock Distribution {Synchrophasors}: {Pairing} {Big} {Data} with
{Analytics} to {Create} {Actionable} {Information}.
\newblock {\em IEEE Power Energy Mag.} {\bf 2018}, {\em 16},~26--34.
\newblock
doi:{\changeurlcolor{black}\href{https://doi.org/10.1109/MPE.2018.2790818}{\detokenize{10.1109/MPE.2018.2790818}}}.

\bibitem[Liu \em{et~al.}(2017)Liu, You, Yao, Cui, Wu, Zhou, Zhao, Liu, and
Liu]{liu_distribution_2017}
Liu, Y.; You, S.; Yao, W.; Cui, Y.; Wu, L.; Zhou, D.; Zhao, J.; Liu, H.; Liu,
Y.
\newblock A {Distribution} {Level} {Wide} {Area} {Monitoring} {System} for the
{Electric} {Power} {Grid}: {FNET}/{GridEye}.
\newblock {\em IEEE Access} {\bf 2017}, {\em 5},~2329--2338, doi:10.1109/ACCESS.2017.2666541.

\bibitem[Bayindir \em{et~al.}(2016)Bayindir, Demirbas, Irmak, Cetinkaya, Ova,
and Yesil]{bayindir_effects_2016}
Bayindir, R.; Demirbas, S.; Irmak, E.; Cetinkaya, U.; Ova, A.; Yesil, M.
\newblock Effects of renewable energy sources on the power system.
\newblock In {Proceedings of the } 2016 {IEEE} {International} {Power} {Electronics} and {Motion}
{Control} {Conference} ({PEMC}), \hl{2016}; pp. 388--393, doi:10.1109/EPEPEMC.2016.7752029.

\bibitem[Farhoodnea \em{et~al.}(2012)Farhoodnea, Mohamed, Shareef, and
Zayandehroodi]{farhoodnea_power_2012}
Farhoodnea, M.; Mohamed, A.; Shareef, H.; Zayandehroodi, H.
\newblock Power quality impact of grid-connected photovoltaic generation system
in distribution networks.
\newblock In {Proceedings of the } 2012 {IEEE} {Student} {Conference} on {Research} and {Development}
({SCOReD}), \hl{2012}; pp. 1--6, doi:10.1109/SCOReD.2012.6518600.

\bibitem[Shafiullah \em{et~al.}(2014)Shafiullah, Oo, Ali, Wolfs, and
Stojcevski]{shafiullah_experimental_2014}
Shafiullah, G.M.; Oo, A.M.T.; Ali, A.B.M.S.; Wolfs, P.; Stojcevski, A.
\newblock Experimental and simulation study of the impact of increased
photovoltaic integration with the grid.
\newblock {\em J. Renew. Sustain. Energy} {\bf 2014}, {\em
6}, doi:10.1063/1.4885105.

\bibitem[Kucuk and Salor(2013)]{kucuk_assessment_2013}
Kucuk, D.; Salor, O.
\newblock Assessment of extensive countrywide electrical power quality
measurements through a database architecture.
\newblock {\em Electr. Eng.} {\bf 2013}, {\em 95}, 1--19.

\end{thebibliography}


\end{document}


