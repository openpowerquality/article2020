\section{Conclusions and Future Directions}
\label{sec:conclusions}

This project has produced both hardware and software for power quality monitoring with a variety of innovations.  Our OPQ Boxes collect frequency, voltage, THD, and transients with high fidelity, and at a cost that is generally 10x to 100x cheaper than current commercial offerings (though these commercial offerings offer a variety of features not available from OPQ Boxes, so the appropriate choice depends upon the needs of the user).

Our sensor network provides real-time, two-way communication between the sensor nodes (OPQ Boxes) and their cloud services, providing several innovative features for power quality monitoring. The OPQ triggering system exploits temporal locality to request high fidelity data from neighboring boxes when one exceeds a threshold, enabling our network to detect and analyze data that would have been unreported by a naive, threshold-based approach. The OPQ Information Architecture enables Actors at the Phenomena Layer to control sensitivity settings of individual OPQ Boxes in order to improve data collection when power quality anomalies have been predicted.

Our sensor network is designed to efficiently use network and storage resources. Two way communication enables OPQ Boxes to send low fidelity summary statistics on power quality at one second intervals, which can be used by cloud-based services to decide whether to request high fidelity data.  Our TTL mechanism implements a "use it or lose it" approach to cloud-based data, which in our pilot study reduced cloud storage requirements by over ten-fold.

Our pilot study has provided evidence that OPQ sensor networks can provide useful new support for understanding grid stability, particularly in grids with distributed, intermittent renewables. This in part because OPQ Boxes are suitable for installation at the residential level: installation requires only an available wall socket and WiFi connection, no electrician or GPS line-of-sight required. Finally, the low cost of OPQ Boxes means mass deployment across a neighborhood is not financially infeasible.

One future goal is to partner with an organization that can enable OPQ Boxes to be produced at volume. We made the OPQ Boxes by hand for the pilot study, and although we have had numerous requests for OPQ Boxes, we do not possess manufacturing capacity.

With manufacturing capacity will come the ability to deploy OPQ sensor networks at higher scale. We believe that a single sensor network can easily scale to hundreds of boxes before bandwidth and processing constraints become an issue. To scale to many thousands of boxes, we believe a federated approach would work in which individual OPQ sensor networks communicate their findings to each other. Those networks could be designed around natural grid boundaries such as substations.

A second goal is to explore ways to combine OPQ sensor networks with more traditional power quality monitoring tools and standards, such as PQDIF. For example, it is possible that OPQ analyses could be improved with access to additional power data such as current, phase angle, and so forth.

A third goal is to build Actors that operate not only on power quality data but also environmental data such as wind, temperature, humidity, and insolation.  These additional data streams could yield valuable mechanisms for creating Predictive Phenomena for power quality anomalies associated with renewable energy sources. Ultimately, such understanding could lead to ways to significantly increase the amount of distributed renewable energy that can be incorporated into electrical grids.

To conclude, a recent paper by Mohsenian et al \cite{mohsenian-rad_distribution_2018} states, "the main challenge is to go beyond manual methods based on the intuition and heuristics of human experts...it is crucial to develop the machine intelligence needed to automate and scale up the analytics on billions of PMU measurements and terabytes of data on a daily basis and in real time." We believe that the OPQ sensor network represents a small step along the path toward that future.



