\section{Related Work}
\label{sec:related-work}

This section explains how OPQ fits into current industry solutions as well as academic research on power quality monitoring and analysis. For the purposes of this review, we exclude utility-side power quality monitoring and analysis systems.

\subsection{Power quality hardware}
\label{sec:commercial-pq-devices}

There exists a very wide variety of power quality hardware devices, including those made by Fluke \cite{fluke_fluke_2020}, Dranetz \cite{dranetz_dranetz_2020}, Elspec \cite{elspec_elspec_2020}, PowerSide (formerly Power Standards Lab) \cite{powerside_powerside_2020}, ACR Systems \cite{acr_acr_2020}, and OpenZMeter \cite{viciana_openzmeter_2018}.

All of the above devices contrast with the OPQ Box in similar ways. First, all of them collect a wider variety of power quality measures than OPQ Box, and most have been certified according to one or more industry standards. Except for OpenZMeter, they are generally designed to support industrial applications, where the goal is to ensure that the power being supplied to a building or plant is of adequate quality, and/or that the machinery in the plant is not degrading power quality. Apart from the PowerWatch monitor, all of them are attached to electrical mains using current transformers. Finally, all of them are designed for "stand alone" operation: each device can independently gather and assess power data.

While the OPQ Box has much more limited functionality, it is designed to be manufactured for approximately \$75, which is 10 to 100 times less expensive than most commercial devices, and similar is cost to OpenZMeter. The most important distinguishing feature of OPQ Box is that it is designed to grid-level, not single point monitoring, and thus incorporates features (such as two-way communication with the cloud, and subthreshold triggering) that are not present in devices intended for "stand alone" capabilities.

\subsection{Power quality software}
\label{sec:commercial-pq-software}

PQView \cite{electrotek_concepts_pqview_2020},  PQSCADA Sapphire \cite{elspec_ltd_pqscada_2016}, PQDIF \cite{sabin_ieee_2020}, and Grid Protection Alliance \cite{grid_protection_alliance_grid_2020}  are examples of software and/or software standards for manipulating power quality data.

The differences between the way OPQ and the above systems store and manipulate power quality data arise from fundamentally different architectural assumptions and the historical background of the technology. PQView and PQSCADA Sapphire are designed to operate in a technology environment consisting of a large number of installed, "stand alone" power quality monitors built by different vendors. Their goal is to aggregate the data collected by these devices, and in order to do so, they depend upon the PQDif standard as a way to obtain power quality data independent of the vendor and device generating it. This results in a kind of "store and forward" process: power quality data is captured and stored on the device, and then periodically bundled into a PQDif file and sent to the database software.

OPQ Cloud, on the other hand, is designed only to support the capabilities of OPQ Boxes. OPQ Boxes, furthermore, have no "stand-alone" capability; they maintain continuous connection to the Internet and upload power quality data to cloud-based services as needed. This means that OPQ implements a very different approach to representing and transmitting data than PQDif. For details on the representation, see the OPQ Data Model, and for details on communication, see the OPQ Protobuf protocol.

\subsection{Research systems}

Di Manno et al \cite{di_manno_user_2015} describes a PQ monitoring system called PiKu. Unlike OPQ, PiKu is designed as a hardware device for sensing power quality that is directly integrated into a PC. Systems with similar architectures include TRANSIENTMETER, described in Da Ponte et al \cite{daponte_transientmeter:_2000}, BK-ELCOM, described in Bilik et al \cite{bilik_modular_2007}, and a system described in Xu et al \cite{xu_distributed_2012}.

There are also research projects based upon leveraging existing monitoring infrastructure. Suslov et al \cite{suslov_distributed_2014} describes a distributed power quality monitoring system based upon existing phasor measurement units installed by utilities. Sayied et al \cite{sayied_power_2013} describes a system designed using existing smart meters. Kucuk et al \cite{kucuk_extensible_2010} describes a similar system for the Turkish National Grid using utility grid monitoring infrastructure.

Mohsenian-Rad et al. \cite{mohsenian-rad_distribution_2018} designed the $\mu$PMU (phase measurement unit) system which provides distributed power quality measurements over power grid distribution systems. The $\mu$PMUs in conjunction with their backend software provide two types of analytics. Descriptive analytics provide information about the types and classifications of power quality issues that are observed within the power distribution grid. Predictive analytics are used to predict future power quality issues. The authors describe their system as providing the ground for for enabling future prescriptive analytics, which is the idea of self-tuning the DSN to prepare for future power quality problems by using a combination of descriptive and predictive analytics. In contrast, as we will discuss in Section \ref{sec:adaptive-optimization}, the OPQ sensor network already supports self-tuning through adaptive optimization.

One research system very similar in spirit to OPQ is FNET \cite{liu_distribution_2017}. Like OPQ, the FNET system consists of custom hardware that monitors the electrical signal from a wall outlet, and uploads data to the cloud for further processing. Unlike OPQ, FNET is designed for monitoring of frequency disturbances, how they propagate across wide area (i.e. nation-wide) utility grids, and, if possible, where the frequency disturbance originated. This means that FNET devices must be synchronized using GPS, and that the data collected consists of frequency and voltage angle. OPQ is designed for more "local" grid analysis, and we are not interested in propagation. As a result, OPQ Boxes are synchronized using NTP rather than GPS, which reduces cost and simplifies installation (OPQ Boxes do not need line of site to a GPS satellite). Finally, FNET hardware appears to support only "one way" communication from device to the cloud, while OPQ Boxes support "two way" communication (from box to cloud, and from cloud to box).
