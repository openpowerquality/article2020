\documentclass[12pt]{letter}
\usepackage{times}
\usepackage[left=1in,top=1in,right=1in,bottom=1in]{geometry}
\signature{Philip Johnson}
\address{Information \& Computer Sciences \\ University of Hawaii \\ Honolulu, HI USA 96822}
\longindentation=0pt
\begin{document}

\begin{letter}{}

{\bf Comments from Reviewer 2:}

{\em (1) In my opinion the paper is too long. Getting through the whole paper is very tiring. It seems that the maximum length of a scientific article (not including appendix) should be 20–22 pages. Maybe some of the descriptions can be moved to the appendix.}

{\em (2) I suggest a typical layout used in the articles: Introduction, Methods, Results and Discussion, Conclusions. This template is also recommended by Journal (see Instructions for Authors).}

{\em (3) Because this is a scientific article, the aims of the work and the research gap should be more clearly highlighted. I also suggest to improve the conclusions, delete the description of the work carried out (it is already well known) and focus on the most important conclusions and the future research. }

{\em (4) The quality of Figures should be corrected, their descriptions, especially axis descriptions are unreadable.}


Fixed.  The issue is incorporating modern web application development as the target domain.  The tech stack associated with modern web application development is complex enough that the semester can easily devolve into ``learning the tech stack", not ``experience software engineering".  Historically, the target domains for project, studio, and flipped pedagogies required only a simple tech stack and so this was not an issue.


\end{letter}
\end{document}
