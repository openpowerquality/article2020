\documentclass[12pt]{letter}
\usepackage{times}
\usepackage[left=1in,top=1in,right=1in,bottom=1in]{geometry}
\signature{Philip Johnson}
\address{Information \& Computer Sciences \\ University of Hawaii \\ Honolulu, HI USA 96822}
\longindentation=0pt
\begin{document}

\begin{letter}{}

{\bf Comments from Reviewer 3:}

{\em (1) Ideas presented in the article are well described, including their advantages and limitations. The introduction covers the basic overview of the topic. It is hard to point out a single specific problem of the article since authors are very precisely defining the structure and following this in the article body (some abbreviations are used before explanation (e.g. THD). Figures 6c and 6d are missing).}

Thanks for the kind words. 




{\em (2) Generally, I am looking for a scientific background of the article – and it is tough to find. Authors have already realized a lot of engineering work (and their presentation is precise); nevertheless, a deep (or deeper) scientific level is missing from my point of view. Anyway, the paper might be useful for future readers and can inspire them in their work. Therefore I recommend it for publishing.}


Fixed.  The issue is incorporating modern web application development as the target domain.  The tech stack associated with modern web application development is complex enough that the semester can easily devolve into ``learning the tech stack", not ``experience software engineering".  Historically, the target domains for project, studio, and flipped pedagogies required only a simple tech stack and so this was not an issue.


\end{letter}
\end{document}
