\documentclass[12pt]{letter}
\usepackage{times}
\usepackage[left=1in,top=1in,right=1in,bottom=1in]{geometry}
\signature{Philip Johnson}
\address{Information \& Computer Sciences \\ University of Hawaii \\ Honolulu, HI USA 96822}
\longindentation=0pt
\begin{document}

\begin{letter}{}

{\bf Comments from Reviewer 3:}

{\em (1) Ideas presented in the article are well described, including their advantages and limitations. The introduction covers the basic overview of the topic. It is hard to point out a single specific problem of the article since authors are very precisely defining the structure and following this in the article body (some abbreviations are used before explanation (e.g. THD). Figures 6c and 6d are missing).}

Thank you for your comments.

As part of the paper reorganization and shortening, Figure 6 has been removed.


{\em (2) Generally, I am looking for a scientific background of the article – and it is tough to find. Authors have already realized a lot of engineering work (and their presentation is precise); nevertheless, a deep (or deeper) scientific level is missing from my point of view. Anyway, the paper might be useful for future readers and can inspire them in their work. Therefore I recommend it for publishing.}

We modified the structure to provide a more traditional scientific format: Introduction, Methods, Related Work, Results and Discussion, Conclusions and Future Directions. We shortened the article from 32 pages to 24 pages. We believe this provides more focus on scientific aspects and results.

\end{letter}
\end{document}
