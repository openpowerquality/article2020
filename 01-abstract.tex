% Abstract (single paragraph, 200 words max, background, methods, results, conclusions. Currently 193 words.)

\abstract{Modern electrical grids are transitioning from a centralized generation architecture to an architecture with significant distributed, intermittent generation.  This transition means that the formerly sharp distinction between energy producers (utility companies) and consumers (residences, businesses, etc) are blurring: end-users both produce and consume energy, making energy management and public policy more complex. The goal of the Open Power Quality (OPQ) project is to design and implement a low cost, distributed power quality sensor network in order to provide useful information about electrical grids (and their stability) in this new era to producers, consumers, researchers, and policy makers.  In 2019, we performed a pilot study involving the deployment of an OPQ sensor network one the University of Hawaii microgrid for three months.  Results of the pilot study provide evidence that OPQ provides a variety of useful services and that the system could be scaled to service larger geographic regions. We conclude that OPQ provides a new and useful approach to power quality monitoring that can be used in conjunction with other technologies .}

% Keywords
\keyword{Power Quality, Open Source, Renewable Energy, Grid Stability}
