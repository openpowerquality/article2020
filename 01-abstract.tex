% Abstract (single paragraph, 200 words max, background, methods, results, conclusions. Currently 193 words.)

\abstract{Modern electrical grids are transitioning from a centralized generation architecture to an architecture which must accomodate distributed, intermittent generation.  This transition also means that the formerly sharp distinction between energy producers (i.e. utility companies) and consumers (residences, businesses, etc) are blurring: residences can now both produce and consume energy, making energy policy more complex. The Open Power Quality (OPQ) project began in 2013 with the goal of designing and implementing a low cost, distributed power quality sensor network in order to provide make useful information about electrical grids available to producers, consumers, and policy makers.  Since then, we have designed low cost hardware devices that monitor power quality  and low-cost cloud-based software services that can economically analyze the data and detect a variety of anomalies. In 2019, we performed a pilot study where 15 OPQ hardware devices were deployed across the University of Hawaii microgrid for three months.  Results of the pilot study provide evidence that OPQ provides a variety of useful monitoring services and that the system could be scaled to service larger geographic regions. We conclude that OPQ provides a new and useful approach to power quality monitoring.}

% Keywords
\keyword{Power Quality, Open Source, Renewable Energy, Grid Stability}
