% Abstract (single paragraph, 200 words max, background, methods, results, conclusions. Currently 193 words.)

\abstract{Modern electrical grids are transitioning from a centralized generation architecture to an architecture with significant distributed, intermittent generation.  This transition means that the formerly sharp distinction between energy producers (utility companies) and consumers (residences, businesses, etc) will blur: customers will both produce and consume energy, making energy management and public policy more complex. The goal of the Open Power Quality (OPQ) project is to design and implement a low cost, distributed power quality sensor network in order to provide make useful information about electrical grids available to producers, consumers, and policy makers.  In 2019, we performed a pilot study where 15 OPQ hardware devices were deployed across the University of Hawaii microgrid for three months.  Results of the pilot study provide evidence that OPQ provides a variety of useful monitoring services and that the system could be scaled to service larger geographic regions. We conclude that OPQ provides a new and useful approach to power quality monitoring.}

% Keywords
\keyword{Power Quality, Open Source, Renewable Energy, Grid Stability}
