\chapter{Conclusions}\label{ch:conclusion}

This dissertation presented the Laha abstract distributed sensor network framework.

Chapter~\ref{ch:introduction} introduced the Laha framework (Section~\ref{sec:laha:-an-abstract-framework-for-adaptively-optimizing-dsns}) and the main problems that the Laha Framework aims to solve, namely the conversion of primitive sensor data into actionable insights (Section~\ref{sec:converting-sensor-data-into-actionable-insights}) and the management of Big Data in relation to DSNs (Section~\ref{sec:big-data-management-in-dsns}). Traditional approaches to DSN optimization were briefly examined (Section~\ref{sec:traditional-approaches-to-dsn-optimization}). This chapter also provided the major claims of the Laha framework (Section~\ref{sec:anticipated-contributions-of-laha}) as well as the major contributions to the field of DSNs (Section~\ref{subsec:anticipated-contributions}).

Chapter~\ref{ch:related-work} examined related work with an emphasis on Big Data and distributed sensor networks (Section~\ref{sec:big-data-and-distributed-sensor-networks}), DSN Big Data management (Section~\ref{sec:distributed-sensor-networks-and-big-data-management}), predictive analytics and forecasting for DSNs (Section~\ref{sec:distributed-sensor-networks-and-predictive-analytics-and-forecasting}), topology and localization (Section~\ref{sec:determining-topology-and-localization}), and triggering optimizations (Section~\ref{sec:optimizations-for-triggering}).

Chapter~\ref{ch:system-design} provided the design details of the Laha framework as well as the design details for the Lokahi and OPQ Laha-compatible reference networks. This chapter included the design of the Laha hierarchy for DSN Big Data Management (Section~\ref{sec:big-data-management}), the design of Phenomena (Section~\ref{sec:phenomena}), design of Laha Actors (Section~\ref{sec:laha-actors:-acting-on-the-laha-data-model}), design of the OPQ reference network (Section~\ref{sec:opq:-a-laha-compliant-power-quality-dsn}), and the design of the Lokahi reference network (Section~\ref{sec:lokahi:-a-laha-compliant-infrasound-dsn}).

Chapter~\ref{ch:evaluation} provided evaluation techniques for determining if the Laha framework is able to meet the goals set in the Introduction chapter. In particular, this chapter examined deployment plans for the OPQ and Lokahi networks (Section~\ref{sec:deploy-laha-reference-implementations-on-test-sites}), data validation strategies (Section~\ref{sec:validate-data-collected-by-laha-deployment}), the evaluation of determining if Laha meets the goals stated in the Introduction chapter (Section~\ref{sec:use-laha-deployments-to-evaluate-the-main-goals-of-the-framework}), and a set of tertiary goals for evaluation (Section~\ref{sec:evaluation-of-tertiary-goals}).

Chapter~\ref{ch:results} provided evidence and results from the Lokahi and OPQ networks that were used to give credence to the goals and contributions outlined in the Introduction chapter. Results were provided for data validation (Section~\ref{sec:ground-truth-analysis}), the generality of the Laha framework (Section~\ref{sec:results-of-generality-of-this-framework}), the ability to convert primitive sensor data into actionable insights (through the Laha level hierarchy and Phenomena (Section~\ref{sec:results-of-converting-primitie-data-into-actional-insights})), tiered Big Data management (Section~\ref{sec:dsn-system-requirements}), and results for the provided tertiary goals (Section~\ref{sec:results-of-tertiary-goals}).

The results showed that Laha is a general framework that can be applied to multiple DSN domains. I showed in the results section that both the OPQ and Lokahi networks were able to meet the stated goals of those networks. In particular, the OPQ network was able to identify distributed PQ signals consisting of transient, voltage, and frequency deviations while the Lokahi network was able to identify infrasonic signals of interest from multiple sources including storms, explosions, and atmospheric disturbances. I showed that the Laha framework is able to convert primitive data into actionable insights through its level hierarchy and also through Phenomena which provide groupings of Incidents and predictive analytic capabilities. I showed that the Laha level hierarchy in conjunction with TTL provides enhanced data management in the form of reducing sensor noise and network resource consumption requirements. Results for TTL of data showed an overall data reduction of close to 96\%. I showed that Phenomena are able to optimize lower levels of the Laha hierarchy which increase its ability to detect sub-threshold Events, detect periodic signals of interest, predict future signals of interest, and reduce data storage requirements which form the basis of Laha's tertiary goals.

\section{Future Directions}\label{sec:future-directions}

The longer I have worked with these networks, the more I have realized that they could be expanded in a multitude of ways.

\subsection{Machine Learning}\label{subsec:machine-learning}
I think the lowest hanging fruit for Laha is to implement an unsupervised machine learning layer. I believe machine learning could be used for triggering, detection, and classification of signals of interest. This is an active area of research within the Lokahi network as we are currently planning to augment our architecture with machine learning. The goals for machine learning within Lokahi are to implement robust detection algorithms using a training set of labeled data collected at our lab and at various national laboratories.

I also believe supervised machine learning could be useful at the Phenomena level, providing models for predicting Events and Incidents and identifying groupings of data. It could be useful to augment Annotation Phenomena with the ability to automatically create new Annotations from past data.

\subsection{Modifying Windows and Thresholds}\label{subsec:modifying-windows-and-thresholds}
To improve the process of creating Events and Incidents, I believe it would be useful to experiment with changing window sizes used to compute low level metrics such as Frequency, THD, and Voltage during temporal network analysis. As shown in the ground truth analysis, the current implementation uses cycle sized windows for computing THD and frequency, but has a cost of added noise. These window sizes could be modified to find an optimum length that minimizes noise, but still accurately reflects the data. By minimizing noise, the system is able to store less data while maximizing system resource allocation for the detection and analysis of signals of interest. As networks scale, this problem becomes more pronounced and noise reduction becomes even more relevant. This is especially true for resource constrained networks which may not have the resources required for storing and filtering data with a low signal-to-noise ratio.

\subsection{More Simulations}\label{subsec:more-simulations}
Although I created a simulation to simulate Laha itself, I believe it would be useful to simulate the power grid as well. Multiple commercial options exist that provide grid simulations. It could be useful to create a copy of the UHM micro-grid in simulation to help fill in some of the missing puzzle pieces about sensor topology and how signals travel through the UHM micro-grid. This would also afford us the opportunity to simulate PQ signals at will instead of waiting for them to arrive. Simulations could allow researchers to more accurately model the sensing field topology. Simulations could also be used to determine optimum sensor placement when sensor availability is low, increasing the chances of identifying target signals of interest. Simulations could also allow us to model the Laha hierarchy in situations where the ground truth is not known or the sensing field topology is unknown.

\subsection{Altering the Laha Level Hierarchy}\label{subsec:altering-the-laha-level-hierarchy}
I would like to experiment with adding and/or combining levels within the Laha hierarchy as described in the ``Discussion of Laha Levels" section. An additional Sensor Measurement Level could be implemented to differentiate between data stored on sensors and data that is stored ``in the cloud". Data stored in the cloud would utilize the same IML level that currently exists, but instead of copying IML data into higher levels, higher levels would simply point to the IML data. An experiment could be constructed that measures the amount of data stored on sensors at any one time versus the amount of data stored in the cloud with respect to raw sensor samples. This experiment would also examine data savings provided by pointing to IML data instead of copying IML data into higher Laha levels.

I believe that Laha is a perfect test bed for data fusion. I would like to integrate multiple data streams into the DSNs to find correlations in the data providing more context for the signals that we observe. For instance, solar production and other environmental data would provide useful data streams for the OPQ network to compare signals against. These new data streams would be provided in new level called the Data Fusion Level (DFL). Specifically, an experiment comparing solar production to PQ utilizing this new level could provide interesting insights into how distributed renewable energy sources directly affect power quality on the grid. With a high availability of solar energy potential, Hawaii makes a perfect test bed for such an experiment. Other data streams that could be fed into this level include cloud coverage and radar data, precipitation data, and general weather data.

\subsection{Enhanced Metric Collection}\label{subsec:enhanced-metric-collection}
I believe Laha could do a better job at collecting metrics about system performance. It would be good to know exactly when data is garbage collected. It would also be useful to collect more memory and system utilization metrics per plugin to determine the performance overhead of individual pieces of analysis.

Future deployments could investigate utilizing more detailed ground truth metrics. The ground truth metrics utilized by OPQ only provided high level trends for voltage, frequency, and THD. It could be useful to have ground truth metrics that include some sort of indication of anomalous PQ events because the UHM ground truth only provided trend data and Events and Incidents were extracted from the trend data by applying thresholds used in the OPQ network. Ground truth data that has a built-in notion of events could be more accurate than determining where the ground truth data should have observed events.

\subsection{Expanded Sensor Coverage}\label{subsec:expanded-sensor-coverage}
Finally, I would like to develop and deploy more sensors for OPQ outside of the UHM micro-grid. It would be useful to discover the interactions in PQ between multiple grids, island wide, and between islands. By having expanded sensor coverage, I believe OPQ could be utilized for solving larger scale problems. For example, an island wide deployment could be useful for accurately determining how distributed intermittent renewable energy sources affect the power grid as a whole and also how renewable energy sources affect individual communities. A state wide deployment between islands could be useful in determining how different utility providers affect PQ in relation to distributed renewable energy sources. A nation wide deployment of OPQ Boxes could provide details about how multiple connected power grids affect power quality and could provide metrics on how PQ signals travel across the grid on a much larger scale. OPQ Boxes could also be deployed in other countries, specifically developing countries, to better understand where PQ issues arise and provide insights into ways to mitigate these PQ problems. Deployments of OPQ Boxes near sensitive electronics (such as server farms) could be used to monitor PQ and its affects on electronic equipment, potentially alerting users to problems before they occur and providing cost savings in terms of reduced hardware maintenance and turnover.

\subsection{Final Thoughts on Future Directions}\label{subsec:final-thoughts-on-future-directions}
Laha enables developers of DSNs to solve the major problems facing modern DSNs. These namely include the collection, storage, management, and analysis of Big Data created from DSNs. These improvements allow us to solve some of the higher level problems facing us today. One such example is utilizing Laha to deal with examining how the integration of large scale intermittent renewable energy sources within the Hawaiian power grids affects PQ and providing measures and metrics to mitigate those problems. Another example includes utilizing Laha to setup early warning systems for tsunamis or volcanic eruptions within the state of Hawaii or abroad. Laha enables the deployment of large scale high volume DSNs for the purpose of solving problems in different domains that would be difficult to solve without Laha.
