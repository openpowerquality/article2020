An OPQ sensor network is designed to be useful to: consumers who want to better understand the performance of their public utilities; utilities who desire a low-cost, easily deployable and re-deployable sensor network to gather information about grid stability that complements their existing infrastructure; to researchers who wish to design improvements to grid management and need a low cost mechanism for measuring the impact of their improvements; and to public policy makers responsible for designing and implementing regulatory frameworks for electrical utilities.  Several key features of our project emerge from these goals.

{\em OPQ Box.} First, our custom power quality monitoring hardware device, called "OPQ Box" can be produced for approximately US\$75. This is from 10 to 100 times less expensive than commercial power quality monitors. This cost differential means, for example, that instead of monitoring power quality at the level of individual buildings, OPQ makes it economically feasible to monitor power quality on each floor or even in each room of a building. Similarly, instead of monitoring power quality at the substation level of the grid, OPQ Boxes make it feasible to deploy dozens or hundreds into a community to obtain fine-grained data about the impact of solar or other renewables as directly perceived by the customer.

{\em Cloud-native information architecture.} Second, our information architecture is "cloud native", which means that OPQ Boxes are not designed for stand-alone, autonomous use, unlike current commercially available power quality monitors. Instead, our hardware boxes and cloud-based services perform real-time, two-way communication over the Internet to determine what power quality data must be gathered and with what fidelity in order to most efficiently produce actionable information. Two-way communication between the OPQ Boxes and cloud services creates the ability to control individual OPQ Boxes based upon the global state of all OPQ Boxes in an OPQ sensor network. As a result, OPQ cloud services can obtain and analyze high fidelity waveform data only when desired, without the prohibitive network overhead of constantly communicating this data from boxes to the cloud. This enables OPQ installations to be simultaneously responsive and scalable.

{\em Plug-in analysis architecture.} Third, all OPQ software services are designed for extensibility and interoperation. Our middleware components (Mauka and Makai) provide a plugin architecture. Our visualization component (View) is built from modular UI elements using React.  This means that OPQ is not a monolithic, closed system with a frozen feature set, but rather a design environment for experimentation with advanced classification and analysis of power quality data.

{\em Open Source.} Fourth, our software and hardware designs and implementations are made available using open source licenses. This means that organizations choosing OPQ do not face the business risk of committing resources to a single vendor with proprietary hardware and software. Our goal is to facilitate the creation of a community of researchers and industry practitioners to replicate, extend, and apply the insights gained from the OPQ system.

While we believe OPQ offers a compelling combination of capabilities, we want to be clear that all sensor network designs involve trade-offs. As we will show below, other commercial and research solutions have features not provided by OPQ that may be important depending upon an organization's specific needs. In some cases, OPQ may not be the appropriate choice. In other cases, a hybrid solution consisting of OPQ in combination with other technologies for grid monitoring and analysis may be most appropriate.

\subsection{Structure of this paper}

The remainder of this paper is organized as follows. Section \ref{sec:architecture} presents an overview of the OPQ system architecture and information architecture. Section \ref{sec:related-work} presents related research and technology.  Section \ref{sec:opq-box} discusses OPQ Box and Section \ref{sec:opq-makai} its triggering system. Section \ref{sec:opq-mauka} describes our analysis services, and Section \ref{sec:opq-view} our user interface.  Section \ref{sec:pilot-study} discusses the pilot deployment of a sensor network and its results. Section \ref{sec:conclusions} concludes with some proposals for future research.
